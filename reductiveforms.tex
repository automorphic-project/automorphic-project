\IfFileExists{stacks-project.cls}{%
\documentclass{stacks-project}
}{%
\documentclass{amsart}
}

% The following AMS packages are automatically loaded with
% the amsart documentclass:
%\usepackage{amsmath}
%\usepackage{amssymb}
%\usepackage{amsthm}

\usepackage{amssymb}

% For dealing with references we use the comment environment
\usepackage{verbatim}
\newenvironment{reference}{\comment}{\endcomment}
%\newenvironment{reference}{}{}
\newenvironment{slogan}{\comment}{\endcomment}
\newenvironment{history}{\comment}{\endcomment}

% For commutative diagrams you can use
% \usepackage{amscd}
\usepackage[all]{xy}

% We use 2cell for 2-commutative diagrams.
\xyoption{2cell}
\UseAllTwocells

% To put source file link in headers.
% Change "template.tex" to "this_filename.tex"
% \usepackage{fancyhdr}
% \pagestyle{fancy}
% \lhead{}
% \chead{}
% \rhead{Source file: \url{template.tex}}
% \lfoot{}
% \cfoot{\thepage}
% \rfoot{}
% \renewcommand{\headrulewidth}{0pt}
% \renewcommand{\footrulewidth}{0pt}
% \renewcommand{\headheight}{12pt}

\usepackage{multicol}

% For cross-file-references
\usepackage{xr-hyper}

% Package for hypertext links:
\usepackage{hyperref}

% For any local file, say "hello.tex" you want to link to please
% use \externaldocument[hello-]{hello}
\externaldocument[introduction-]{introduction}
\externaldocument[representationtheory-]{representationtheory}
\externaldocument[representations-compact-]{representations-compact}
\externaldocument[liegroups-general-]{liegroups-general}
\externaldocument[liestructure-]{liestructure} 
\externaldocument[vermamodules-]{vermamodules}
\externaldocument[algebraicgroups-]{algebraicgroups}
\externaldocument[reductiveforms-]{reductiveforms}
\externaldocument[galoiscohomology-]{galoiscohomology}
\externaldocument[representations-local-]{representations-local}
%\externaldocument[gKmodules-]{gKmodules}
%\externaldocument[asymptotics-]{asymptotics}
\externaldocument[plancherel-]{plancherel}
\externaldocument[discreteseries-]{discreteseries}
\externaldocument[automorphicspace-]{automorphicspace}
%\externaldocument[harmonicanalysis-]{harmonicanalysis} 
\externaldocument[automorphicforms-]{automorphicforms}
%\externaldocument[periods-]{periods}
%\externaldocument[traceformulalocal-]{traceformulalocal}
%\externaldocument[traceformulaglobal-]{traceformulaglobal}
%\externaldocument[arithmetic-]{arithmetic}
%\externaldocument[geometric-]{geometric}
\externaldocument[fdl-]{fdl}
\externaldocument[index-]{index}

% Theorem environments.
%
\theoremstyle{plain}
\newtheorem{theorem}[subsection]{Theorem}
\newtheorem{proposition}[subsection]{Proposition}
\newtheorem{lemma}[subsection]{Lemma}

\theoremstyle{definition}
\newtheorem{definition}[subsection]{Definition}
\newtheorem{example}[subsection]{Example}
\newtheorem{exercise}[subsection]{Exercise}
\newtheorem{situation}[subsection]{Situation}

\theoremstyle{remark}
\newtheorem{remark}[subsection]{Remark}
\newtheorem{remarks}[subsection]{Remarks}

\numberwithin{equation}{subsection}

% Macros
%
\def\lim{\mathop{\rm lim}\nolimits}
\def\colim{\mathop{\rm colim}\nolimits}
\def\Spec{\mathop{\rm Spec}}
\def\Hom{\mathop{\rm Hom}\nolimits}
\def\SheafHom{\mathop{\mathcal{H}\!{\it om}}\nolimits}
\def\SheafExt{\mathop{\mathcal{E}\!{\it xt}}\nolimits}
\def\Sch{\textit{Sch}}
\def\Mor{\mathop{\rm Mor}\nolimits}
\def\Ob{\mathop{\rm Ob}\nolimits}
\def\Sh{\mathop{\textit{Sh}}\nolimits}
\def\NL{\mathop{N\!L}\nolimits}
\def\proetale{{pro\text{-}\acute{e}tale}}
\def\etale{{\acute{e}tale}}
\def\QCoh{\textit{QCoh}}
\def\Ker{\text{Ker}}
\def\Im{\text{Im}}
\def\Coker{\text{Coker}}
\def\Coim{\text{Coim}}

\def\eqref #1{(\ref{#1})}
\newcommand{\sslash}{\mathbin{/\mkern-6mu/}}


% OK, start here.
%
\begin{document}

\title{Forms and covers of reductive groups, and the $L$-group}


\maketitle

\phantomsection
\label{section-phantom}

\tableofcontents



\section{Classification of reductive groups over a separably closed field}
\label{section-classification-reductive}


For this section we assume $k$ to be separably closed (unless otherwise noted). [For now, this section is missing a lot of results, as one needs to establish the analogs of results that we proved for semisimple Lie algebras in characteristic zero, for reductive groups.]

We saw in Theorem \ref{algebraicgroups-theorem-diagonalizable-equivalence} a simple combinatorial description for diagonalizable groups, and we would like to have a similar description for more general reductive groups. This is not possible in the sense of getting an equivalence of categories\footnote{There is a good reason for it: To get an equivalence of categories one must consider the category of all $G$-representations, cf.\ Tannaka-Krein duality. For diagonalizable groups this category is described easily in terms of combinatorial data, this is no longer the case for other groups.}, but at least we can fully describe the isomorphism classes this way.


Let $G$ be reductive (over $k$), and let $T$ be a maximal torus in $G$. Let $X^\bullet(T)$, $X_\bullet(T)$ be the character and cocharacter groups of $T$. The adjoint action of $T$ on $\mathfrak g$ is semisimple, and we have a decomposition 
$$\mathfrak g = \mathfrak g_0 \oplus \bigoplus_{\alpha\in\Phi} \mathfrak g_\alpha,$$
where the $\alpha$'s, here, are eigencharacters $\alpha: T\to \mathbf G_m$. 

By Proposition \ref{algebraicgroups-proposition-Cartan-reductive}, $\mathfrak g_0$ is just the Lie algebra of $T$, however, \emph{to accommodate the case of positive characteristic}, we will not be using $\mathfrak t$ to denote this Lie algebra, but the real vector space
$$ \mathfrak t = X_\bullet(T)\otimes \mathbb R$$
(and, similarly, $\mathfrak t^*= X^\bullet(T)\otimes \mathbb R$). For real algebraic groups, this $\mathfrak t^*$ can be identified with the dual of the Lie algebra,  by the map that assigns to any character $\chi$ its differential $d\chi$ at the identity. 

\begin{definition}[Root datum] 
\label{definition-root-datum}
 A {\it root datum} is a quadruple $(X,\Phi, \check X,\check\Phi)$, where $X, \check X$ are two lattices (finitely generated, torsion-free abelian groups) in duality, $\Phi\subset X$ and $\check\Phi\subset\check X$ are finite subsets, such that there exists a bijection $\Phi \ni \alpha\leftrightarrow \check\alpha\in\check\Phi$, satisfying:
\begin{enumerate}
\item $\left< \alpha,\check\alpha\right>=2$;
\item the endomorphisms of $X,\check X$ defined by $w_\alpha(x):= x-\left< x, \check\alpha\right> \alpha$, $w_{\check\alpha}(\check x)= \check x - \left< \alpha,\check x\right> \check\alpha$ preserve $\Phi$ and $\check\Phi$.
\end{enumerate}

A {\it based root datum} is a root datum as above, together with a choice of positive roots $\Phi^+\subset \Phi$ (in the sense of Definition \ref{liestructure-definition-based-root-system}).
\end{definition}

\begin{remarks}
\label{remarks-root-datum}

 \begin{enumerate}
  \item The last axiom is equivalent to: the endomorphisms $w_\alpha$ preserve $\Phi$ and generate a finite group (the Weyl group $W$).
  \item The condition $\left< \alpha,\check\alpha\right>=2$ characterizes the bijection $\Phi\leftrightarrow \check\Phi$, so it need not be part of the data.
  \item The $\mathbb R$-span of $\Phi$ in $X^*\otimes \mathbb R$, together with $\Phi$ and the Weyl group of automorphisms generated by the $w_\alpha$'s, forms a root system, as can be easily verified; hence the definition of based root datum in terms of based root systems.
 \end{enumerate}

\end{remarks}


We now define the appropriate notion of morphisms between root data.

\begin{definition}
\label{definition-isogeny-root-data}
 An {\it isogeny} of root data $(X,\Phi, \check X,\check\Phi) \to (X',\Phi', \check X',\check\Phi')$ is a homomorphism $f:X\to X'$ with the following properties:
 \begin{enumerate}
  \item $f$ is injective, and with finite cokernel --- hence, so is its adjoint $f^*:\check X'\to \check X$;
  \item $f$ and $f^*$ induce bijections between the subsets of roots and coroots, respectively.
 \end{enumerate}
 \end{definition}





\begin{proposition}
\label{proposition-simple-reflections}
 Given a connected reductive group $G$ and a maximal torus $T$ with set of roots $\Phi\subset X^*(T)$, and given $\alpha\in \Phi$, consider the subtorus $T_\alpha=\text{ker}(\alpha)^\circ\subset T$. Let $L_\alpha$ be the centralizer of $T_\alpha$ (which is connected by Proposition \ref{algebraicgroups-proposition-centralizers-tori-connected}, and $L_\alpha'$ its derived subgroup. Then $L_\alpha'$ is isomorphic to $\text{SL}_2$ or $\text{PGL}_2$, and therefore there is a unique cocharacter $\check\alpha:\mathbf G_m\to T\cap L_{\alpha}'$ with $\left< \alpha, \check\alpha\right>=2$. If $w_\alpha$ is the nontrivial element of the Weyl group of $T$ inside of $L_\alpha'$, then the elements $w_\alpha$ generate the full Weyl group of $T$ in $G$, $W = \mathcal N_G(T)/T$.
\end{proposition}

\begin{proof}
 Omitted.
\end{proof}

\begin{definition}
 \label{definition-coroots-group}
In the notation of Proposition \ref{proposition-simple-reflections}, the elements $\check\alpha\in X_*(T)$ associated to the roots $\alpha\in X^*(T)$ are the {\it coroots} of the torus $T$ in $G$.
\end{definition}


\begin{definition}
\label{definition-central-isogeny}
 Let $(G,T)$, $(G',T')$ be two pairs consisting of a connected reductive group and a maximal torus. A {\it central isogeny} $(G',T')\to (G,T)$ is a morphism $G'\to G$ sending $T'\to T$, surjective, with finite kernel, and inducing isomorphisms between the root spaces of $T'$ and $T$ in $\mathfrak g'$, resp.\ $\mathfrak g$. A central isogeny $G'\to G$ is a morphism  which induces a central isogeny $(G',T')\to (G,T)$ for some maximal tori $T', T$ (or, equivalently, for any maximal torus $T'$, and $T$ the image of $T'$). 
\end{definition}

\begin{remark}
 \label{remark-on-central-isogenies}
A surjective morphism $(G',T')\to (G,T)$ with finite kernel gives rise to an injective map $X^*(T)\to X^*(T')$ with finite cokernel, and a bijection between the roots of $T$ on $G$ and the roots of $T'$ on $G'$. The requirement that the map induce an isomorphism between root spaces: $\mathfrak g_\alpha'\to \mathfrak g_\alpha$ is automatically satisfied in characteristic zero, but not in positive characteristic. For example, if $G$ is defined over a finite field with $q$ elements, the Frobenius morphism $F_q$ (see the proof of Theorem \ref{algebraicgroups-theorem-maximal-tori-exist}) restricts to the Frobenius morphism on the subgroups $\mathfrak g_\alpha\simeq \mathbf G_a$, which is not an isomorphism.
\end{remark}


Now we define some categories of reductive groups with extra data that will be used in the classification.

\begin{definition}
 \label{definition-reductive-categories}
We let $Red$ be the category whose objects are reductive groups $G$ over $k$, and whose morphisms are central isogenies $G'\to G$. We let $Red_T$ be the category whose objects are pairs $(G\supset T)$ of reductive groups with maximal tori, and whose morphisms are central isogenies of pairs $(G',T')\to (G,T)$. We let $Red_{B,T}$ be the category whose objects are triples $(G\supset B\supset T)$ of reductive groups with Borel subgroups and maximal tori thereof, and whose morphisms are central isogenies of pairs $(G',T')\to (G,T)$ sending $B'$ to $B$.

We let $RD$ be the category whose objects are root data $(X,\Phi, \check X,\check\Phi)$ and whose morphisms are central isogenies. We let $RD^+$ be the category whose objects are root data, together with a choice $\Phi^+\subset \Phi$ of positive roots.
\end{definition}

\begin{definition}
\label{definition-split-nonsplit}
 If $G$ is a connected reductive group over a field $k$ which is not necessarily separably closed, it is said to be {quasisplit} if there exists a Borel subgroup over $k$, and {\it split} if there exists a maximal torus which is split over $k$ (hence also a Borel subgroup). 
\end{definition}



\begin{theorem}[Classification over the separable closure]
\label{theorem-classification-reductive}
 Assume $k$ separably closed. Given a connected reductive group $G$ and a maximal torus $T$, the quadruple $\Psi(G,T)=(X^*(T), \Phi, X_*(T), \check\Phi)$, where $\Phi,\check\Phi$ denote, respectively, the sets of roots and coroots of $T$ in $G$, is a root datum.
 
 The assignment $(G,T)\mapsto \Psi(G,T)$ is a functor $Red \to RD^{\text{op}}$, in the notation of Definition \ref{definition-reductive-categories}.
 
 This functor is a bijection on isomorphism classes of objects, and every morphism $\Psi(G,T)\to \Psi(G',T')$ in $RD$ lifts to a morphism (i.e., central isogeny) $(G',T')\to (G,T)$, uniquely up to precomposing with conjugation by elements of $T'$, or post-composing with conjugation by elements of $T$.
 
 More generally, the same statement holds if $k$ is not separably closed, but we consider the full subcategory of such pairs $(G,T)$ where $T$ (and, hence, $G$) is split over $k$.
\end{theorem}

\begin{proof}
 Omitted. See \cite{Springer-Corvallis} for more details and references.
\end{proof}




We will now see two variants of this theorem: One, where we choose more data in order to rigidify the functor, and another, where we make no choices (i.e., we do not choose tori).


If, in the context of Theorem \ref{theorem-classification-reductive}, we choose a Borel subgroup $B\supset T$, it gives rise to a based root datum, with $\Phi^+$ being the set of roots appearing in the Lie algebra of $B$. We will generally denote by $\Delta$ the subset of simple roots.

\begin{definition}
 \label{definition-pinning}
A {\it pinning} of a triple $(G, B, T)$ consisting of a connected reductive group $G$, a Borel subgroup,  and a maximal torus $T$ is a choice of isomorphisms $p_\alpha:\mathfrak g_\alpha \simeq \mathbf G_a$, for every simple root $\alpha\in \Delta$.

Given a pair $(G,B)$ of a connected reductive group and a Borel subgroup, whose unipotent radical we will denote by $N$, an {\it algebraic Whittaker datum} is a homomorphism $\ell:N\to \mathbf G_a$ whose differential is nontrivial on every simple root subspace $\mathfrak g_\alpha$. (In particular, its differential induces a pinning; vice versa, given a pinning as above, there is a unique such $\ell$ with $d\ell|_{\mathfrak g_\alpha} = p_\alpha$.)

We let $Red_{\text{PIN}}$ denote the category whose objects are pinned reductive groups, and whose morphisms are central isogenies preserving the Borel, the torus, and the pinning.
\end{definition}

Pinnings allow us to rigidify the morphisms lifted from morphisms of (based) root data:

\begin{theorem}
\label{theorem-classification-with-pinning}
The assignment $(G, B, T, P) \to \Psi^+(G,B,T)$, where $\Psi^+(G,B,T)$ denotes the based root datum consisting of $\Psi(G,T)$ (notation as in Theorem \ref{theorem-classification-reductive}) with $\Phi^+$ the positive roots associated to $B$, is an equivalence of categories: $Red_{\text{PIN}}\to RD^+$.
\end{theorem}

\begin{proof}
 Omitted. See, again, \cite{Springer-Corvallis}. 
\end{proof}

Finally, a version of the classification for morphisms $G'\to G$, without any choices: Recall that to any reductive group $G$, we have associated its (universal) Cartan $\mathbb A^G$, endowed with sets of roots and positive roots $\Phi^+\subset \Phi\subset X^*(\mathbb A^G)$. (Recall that in this section we are assuming the field to be separably closed.) We will denote by $\Psi^+(\mathbb A^G)$ this based root system.

\begin{lemma}
 \label{lemma-Cartan-functorial}
The association $G\to \mathbb A^G$ is functorial in the category $Red$ of reductive groups with central isogenies.
\end{lemma}

\begin{proof}
 If $f:G\to G'$ is a central isogeny, it sends a Borel subgroup $B\subset G$ onto a Borel subgroup $B'\to G'$, inducing morphisms of their reductive quotients: $\mathbb A^G=B/N\to \mathbb A^{G'}=B'/N'$. If we choose another Borel subgroup $B_1\subset G$, there is a $g\in G$ with $g B g^{-1} = B_1$, hence $f(g) B' f(g)^{-1} = B_1':=f(B_1)$, and the morphism $\mathbb A^G\to \mathbb A^{G'}$ induced from $B_1\to B_1'$ is the same as before, given how we have identified $\mathbb A^G$ as the quotient of \emph{any} Borel subgroup.
\end{proof}


\begin{theorem}
\label{theorem-classification-with-universal-Cartan}
The assignment $G \to \Psi^+(\mathbb A^G)$ is a functor $Red\to (RD^+)^{\text{op}}$, in the notation of Definition \ref{definition-reductive-categories}. It induces a bijection on isomorphism classes of objects, and for any morphism $\Psi^+(\mathbb A^{G}) \to \Psi^+(\mathbb A^{G'})$ there is a morphism $G'\to G$, unique up to inner automorphisms.
\end{theorem}

\begin{proof}
 This follows from Theorem \ref{theorem-classification-reductive}, and the conjugacy of Cartan subgroups.
\end{proof}





\subsection{Simply connected and adjoint groups}
\label{subsection-simply-connected-adjoint}

\begin{definition}
 \label{definition-sc-adjoint-rootdatum}
Let $(X,\Phi,\check X,\check\Phi)$ be a root datum, let $R\subset X$, $\check R\subset \check X$ be the subgroups spanned by the roots, resp.\ coroots, and let $P = \check R^*$, $\check P = R^*$ the dual lattices. 

The root datum is called {\it semisimple} if $\mathcal R$ is of finite index in $X$. 

Assume this to be the case, so that we have containments with finite quotients $P\supset X\supset R$ and $\check P\supset\check X\supset \check R$. We say that the root datum  is {\it simply-connected} if $X=\mathcal P$, equivalently $\check X=\check R$, and {\it adjoint} if $X=\mathcal R$, equivalently $\check X =\check P$. 
\end{definition}

\begin{definition}
 \label{definition-sc-adjoint-group}
A reductive group is {\it adjoint} if it has trivial center, and {\it simply connected} if it admits no (connected) central isogenies.
\end{definition}

The relationship between this notion of being simply connected, and the algebrogeometric one, will become clear in Proposition  \ref{proposition-sc-adjoint-group} and Remark \ref{remark-sc}.




\begin{proposition}
 \label{proposition-sc-adjoint-group}
Let $(G,T)$ be a connected reductive group with a maximal torus over a field $k$, and let $\Psi(G,T)$ be the associated root datum.
\begin{enumerate}
\item $G$ is semisimple iff $\Psi(G,T)$ is semisimple.
\item $G$ is adjoint iff $\Psi(G,T)$ is adjoint.
\item $G$ is simply connected (in the sense of Definition \ref{definition-sc-adjoint-group} iff $\Psi(G,T)$ is simply connected. 
\item In characteristic zero, $G$ is simply connected as a scheme (or, equivalently, $G(\mathbb C)$ is simply connected as a topological space when $k=\mathbb C$) iff $\Psi(G,T)$ is simply connected.
\end{enumerate}

\end{proposition}

\begin{proof}
\begin{enumerate}
 \item A reductive group is semisimple iff its center is finite. The center belongs to $T$, and coincides with the common kernel of all roots. In other words, in terms of the equivalence of diagonalizable groups and finitely generated abelian groups (Theorem \ref{algebraicgroups-theorem-diagonalizable-equivalence}), the character group of the center is the cokernel of the inclusion $R\to X$. Hence, it is finite iff $R$ is of finite index in $X$.
 
 (Notice that the center is not necessarily reduced; for example, the center of $\text{SL}_p$ in characteristic $p$ is the non-reduced group scheme $\mu_p$ of $p$-th roots of unity.)
 
 \item Continuing along the same lines, the center is trivial iff $R=X$.
 
 \item A central isogeny $G'\to G$ which is not an isomorphism, mapping some maximal tori $T'\to T$,  induces an isogeny $\Psi(G,T)\to \Psi(G',T')$, which is completely determined by the map of cocharacter groups $X_*(T')\to X_*(T)$. This map is injective, of finite cokernel, and has to preserve coroot lattices, so if $X_*(T)$ is equal to the coroot lattice, the map is an isomorphism.
 
 \item By \cite[Theorem 1]{Brion-Szamuely}, if $p$ is the characteristic exponent of the field, every prime-to-$p$ \'etale Galois cover $G'\to G$ is a central isogeny. In particular, in characteristic zero, $G$ is simply connected in the sense of Definition \ref{definition-sc-adjoint-group} iff it is simply connected in the sense of \'etale topology. Moreover, if $k=\mathbb C$, the \'etale fundamental group is the profinite completion of the topological fundamental group of $G(\mathbb C)$. 
 
\end{enumerate}

\end{proof}


\begin{remark}
\label{remark-sc} 
In positive characteristic, the \'etale fundamental group is always infinite, for a smooth affine scheme $X= \text{Spec}(R)$ of positive dimension. For example, we have the Artin--Schreier $\mathbb Z/p$-covers, $X'=\text{Spec} R[y]/(y^p-y-f)$, if $f\in R$ is chosen appropriately.
\end{remark}





\subsection{Automorphisms} 
\label{subsection-automorphisms}

Given an automorphism $f:X\to X$ of an affine variety $X$ over a field $k$, its defining homomorphism $f^*:k[X]\to k[X]$ restricts to an automorphism $V\to V$ on a generating, finite-dimensional subspace $V$ of $k[X]$. Therefore, the automorphism group $\text{Aut}(X)$ has a natural structure as the ind-algebraic group
$$ \underset{\underset{V}\to}\lim \text{Aut}(X,V),$$
where $V$ runs over all finite-dimensional, generating subspaces of $k[X]$, and $\text{Aut}(X,V)\subset \text{GL}(V)$ is the subgroup of those automorphisms of $V$ (as a vector space) which induce an automorphism of $X$ (as an algebraic variety).

\begin{definition}
 \label{definition-automorphism-group}
The {\it group of inner automorphisms}, $\text{Inn}(G)$, is the image of the natural morphism $G\to \text{Aut}(G)$ given by the conjugation action of $G$ on itself. It is also called the {\it adjoint group} of $G$, and denoted by $G_{\text{ad}}$. 

The quotient $\text{Aut}(G)/\text{Inn}(G)$ is the group of {\it outer automorphisms} of $G$.
\end{definition}

The kernel of the map $G\to \text{Aut}(G)$ is the center of $G$, so the group $G_{\text{ad}}=\text{Inn}(G)$ is the quotient of $G$ by its center.

For reductive groups the automorphism group is of finite type. This follows from the fact that, by the classification of reductive groups in terms of root data, the outer automorphism group is the group of automorphisms of based root data.

\begin{proposition}
 \label{proposition-automorphism-sequence}
The functor from reductive groups to based root data of Theorem \ref{theorem-classification-with-universal-Cartan} gives rise to a short exact sequence of (ind-)algebraic groups:
\begin{equation}
 \label{equation-Inn-Aut-Out}
0\to \text{Inn}(G)\to \text{Aut}(G)\to \text{Aut}\Psi^+(\mathbb A^G) \to 0.
\end{equation}
In particular, $\text{Out}(G)=\text{Aut}\Psi^+(\mathbb A^G) $. 

Moreover, any pinning on $G$ (Definition \ref{definition-pinning}) gives rise to a splitting $\text{Out}(G)\to \text{Aut}(G)$, characterized by the fact that its image consists of the automorphisms preserving the pinning.
\end{proposition}

\begin{proof}
 Applying the functor of Theorem \ref{theorem-classification-with-universal-Cartan} to automorphisms of $G$, we get a homomorphism
 $\text{Aut}(G) \to \text{Aut}\Psi^+(\mathbb A^G)$. By Theorem \ref{theorem-classification-with-universal-Cartan}, this homomorphism is surjective, and its kernel is precisely the group of inner automorphisms. 
 
 The existence of a unique splitting that fixes a given pinning follows from Theorem \ref{theorem-classification-with-pinning}.
\end{proof}



\section{Forms and Galois descent}
\label{section-forms-Galois-descent}

\begin{definition}
 \label{definition-forms}
Let $G$ be a linear algebraic group over a field $k$ or over an extension $L/k$. A {\it form} of $G$ over $k$ is a linear algebraic group $G'$ over $k$, such that $G_L\simeq G'_L$.
\end{definition}


The goal of this section is to explain how isomorphism classes of forms of $G$ are described by the Galois cohomology set $H^1(\Gamma, \text{Aut}(G_{k^s}))$, where $\Gamma = \text{Gal}(k^s/k)$, the Galois group of the separable closure of $k$. There is nothing special about classification of linear algebraic groups here --- the same arguments apply to any category of objects that satisfy \emph{effective descent}. We introduce these notions in some generality, but not the full generality of faithfully flat descent (see \cite[Tag 0238]{stacks-project}).

Let $Aff_k$ be the category of affine schemes over the field $k$. A \emph{k-groupoid} is a category with a functor $\mathcal F\to Aff_k$ satisfying two axioms, which can be summarized as ``base change exists'', and ``fiber categories are groupoids''. To formulate them, denote by $\mathcal F_U$ the full subcategory living over an object $U\in Aff_k$. The axioms are:
\begin{enumerate}
 \item Given $U\in Aff_k$, $u\in \mathcal F_U$, and $f:V\to U$ a morphism, there exists a morphism $\tilde f:v\to u$ in $\mathcal F$ lying over $f$.
 \item For any pair of morphisms $\tilde f: u\to z$, $\tilde g: v\to z$ in $\mathcal F$, lying over $f:U\to Z$ and $g:V\to Z$ in $Aff_k$, and any $h: U\to V$ with $f = g\circ h$, there exists a unique lift $\tilde h: u\to v$ with $\tilde f = \tilde g \circ \tilde h$.
\end{enumerate}

If $v$ is as in the first axiom, we will feel free to write $v=u_V$, i.e., as a base change of $u$. Notice that, for every other choice $v'$, the second axiom provides a canonical isomorphism $v'\to v$, corresponding to the identity map on $V$. For $f: V\to U$ and $u\in \mathcal F_U$, one has a unique morphism $u_V\to u'_V$ over any morphism $\phi: u\to u'$; we will be denoting that by $f^*\phi$. For a composition $W\to V\to U$, we will feel free to identify $(u_V)_W$ with $u_W$ --- see \cite[Tag 02XN]{stacks-project} for a clarification of these issues. 

Now consider a Grothendieck topology on $Aff_k$, turning it into a site. For notational simplicity, we will represent every cover as a single morphism $U\to X$, instead of a family of morphisms. The groupoid $\mathcal F \to Aff_k$ is called a \emph{prestack} (of groupoids) if, for any $U\in Aff_k$, and any $x,y \in \mathcal F_U$, the presheaf $V\mapsto \text{Mor}(x_V,y_V)$ is a sheaf on the site $Aff_U$, and it is called a \emph{stack} (of groupoids) if \emph{descent data are effective}. 

These are two conditions that need to be formulated for arbitrary covers, but we just explain their meaning for the case of covers corresponding to finite Galois extensions $L/k$ with Galois group $\Gamma$; the generalization to arbitrary $U$ and arbitrary covers is straightforward, see \cite[Tag 0268]{stacks-project}. 

Denote $U=\text{Spec}(L) \to X=\text{Spec}(k)$, and notice that there is a canonical isomorphism $U\times_X U \simeq \Gamma\times  U$, equivariant with respect to the action of $\Gamma$ on the first copy, taking $U\times \{1\}$ to the diagonal. A \emph{descent datum} for $L/k$ consists of an object $u\in \mathcal F_U$, together with (iso)morphisms $\phi_\gamma: \gamma u \to u$ (over the \emph{identity} of $U$), where $\gamma u$ is the base change of $u$ via the map $\gamma^{-1}: U\to U$, and the $\phi_\gamma$'s are required to satisfy the cocycle condition $\phi_{\gamma_1 \gamma_2} = \phi_{\gamma_1} \circ \gamma_1^* \phi_{\gamma_2}$. Equivalently, a descent datum is an isomorphism $\phi: u_1 \to u_2$, where $u_i$ is the pullback via the $i$-th projection $U\times_X U\to U$, and the isomorphism is required to satisfy the natural cocycle condition with respect to the triple product $U\times_X U\times_X U$.

Let $\mathcal F_{L/k}$ denote the category of descent data $(u, \phi)$. (The morphisms in this category are the obvious ones, morphisms of the $u$'s commuting with the $\phi$'s.) There is a canonical functor 
\begin{equation}
\label{equation-descent} 
\mathcal F_k \to \mathcal F_{L/k} 
\end{equation}
taking an object $x\in \mathcal F_k = \mathcal F_X$ to its base change $x_L$, together with the descent data induced from the Galois automorphisms of $U=\text{Spec}(L)$ over $k$. (Notice that a morphism $\gamma u \to u$ over the identity in $U$ is equivalent to a morphism $u\to u$ over the map $\gamma: U\to U$.)

Then, $\mathcal F_k$ satisfies \emph{effective descent} (together with the prestack condition) with respect to the Galois cover $L\to k$, if the functor \eqref{equation-descent} \emph{is an equivalence}. Here are two basic examples where it happens:

\begin{example}
 \label{example-descent-vectorspaces}
The category $\mathcal F_k$ of vector spaces over $k$ satisfies effective descent with respect to any separable extension $L/k$. (It can be extended to a $k$-stack $\mathcal F$ by considering vector bundles over an arbitrary basis.)
\end{example}

\begin{example}
\label{example-descent-algebras}
The category $\mathcal F_k$ of algebras over $k$ satisfies effective descent with respect to $L/k$. Indeed, let $B$ be an $L$-algebra. The descent datum amounts to an isomorphism $\phi: B\otimes L \simeq L \otimes B$ (tensor products over $k$), and the cocycle condition states that the triangle consisting of $B\otimes L \otimes L$, $L\otimes B\otimes L$ and $L\otimes L\otimes B$, applying $\phi$ to get isomorphisms between them, commutes. We then take our algebra $A$ over $k$ to consist of those sections of $B$ whose pullbacks to $L\otimes B \overset{\phi}\simeq B\otimes L$ (geometrically, in the notation above: whose pullbacks to $U\times_X U$ under both projections) coincide, i.e., $A=\text{ker}(B\to L\otimes B)$, where the map is given by $b\mapsto 1\otimes b - \phi(b\otimes 1)$. One checks that $A$ is a $k$-algebra, and $B \simeq A \otimes L$ compatibly with descent data, see \cite[Tag 0244]{stacks-project}.
\end{example}

\begin{example}
 \label{example-descent-quasiprojective}
Galois descent is also satisfied for any quasi-projective scheme, see \cite[\S 6.2, Example B]{Neron}.
\end{example}

Example \ref{example-descent-algebras} is the basic one for our purposes. It immediately extends to groupoids of affine varieties with extra structure given by ``closed'' conditions, such as algebraic groups. 


Finally, descent data can be parametrized by torsors for the automorphism group:

\begin{proposition}
 \label{proposition-descent-torsors}
Given $(u, \phi) \in \mathcal F(L/k)$, let $A=\text{Aut}_{\mathcal F_L}(x)$, considered as a $\Gamma$-module (possibly non-abelian) with action ${\gamma a}:= \phi_\gamma\circ a \circ \phi_\gamma^{-1}$. Then, we have a natural equivalence of of categories
$$ \mathcal F(L/k)/u \xrightarrow\sim A-\text{Tors}^\Gamma,$$
where $\mathcal F(L/k)/u $ denotes the category of descent data $(u', \phi')$ with $u'\simeq u$, and $A-\text{Tors}^\Gamma$ denotes the category of (set-theoretic) right $A$-torsors $T$ with a compatible left $\Gamma$-action, i.e., ${^\gamma(t\cdot a)} = {^\gamma t} \cdot {^\gamma a}$ for all $\gamma\in\Gamma, a\in A$, and $t\in T$.

The functors are: 
\begin{itemize}
 \item $(u',\phi')\to \text{Isom}(u, u')$, and
 \item $T \to (u'= T\times^A u, \phi')$, where $\phi' = \{\phi'_\gamma\}$ is obtained by acting diagonally on $T$ (via the given Galois action) and on $u$ (via $\phi$).  
\end{itemize}

\end{proposition}

\begin{proof}
Left to the reader. 
\end{proof}



If $\Gamma$ is a finite group, and $A$ is a (possibly non-abelian) group with a $\Gamma$-action, then isomorphism classes of $A$-torsors with a $\Gamma$-action are naturally parametrized by the \emph{1st cohomology set} $H^1(\Gamma, A)$, which is the pointed set of $A$-orbits on the set of $1$-cocycles $Z^1(\Gamma, A) = \{c:\Gamma\to A| c(\gamma_1 \gamma_2) = c(\gamma_1) \cdot {^{\gamma_1}c(\gamma_2)}\}$, where $A$ acts by twisted conjugation $(a\cdot c)(\gamma) = a c(\gamma) {^\gamma a}^{-1}$. The parametrization takes a cocycle $c$ to the $A$-torsor $T$ which can be identified with $A$ as a right-$A$-set, but with Galois action twisted by $c$, i.e., if we let $a'$ be the element of $T$ corresponding to an element $a$ of $A$, then ${^\gamma a'} = c(\gamma) \cdot {^\gamma a} $. 


Thus, we obtain:

\begin{proposition}
 \label{proposition-descent-cohomology}
The equivalence of \ref{proposition-descent-torsors} induces an isomorphism of pointed sets:
$$ (\mathcal F(L/k)/u)/\tilde{ }\xrightarrow\sim H^1(\Gamma, A),$$
where the left-hand side denotes isomorphism classes of descent data for $u$ over $L/k$, with the chosen descent datum as the base point.
\end{proposition}

\begin{proof}
 One just needs to check that isomorphism classes of $\Gamma$-equivariant $A$-torsors are indeed parametrized by $H^1(\Gamma, A)$. Left to the reader.
\end{proof}



\begin{remark}
 \label{remark-torsors-base-point}
Notice that, although Proposition \ref{proposition-descent-torsors} classifies a family of descent data depending only on the isomorphism class of $u$, the classification, and the Galois structure on the automorphism group $A$, depend on the chosen descent structure on $u$. This choice makes the isomorphism classes of descent data into a pointed set, which corresponds to the distinguished point (the class of the trivial torsor) in $H^1(\Gamma, A)$.
\end{remark}




\section{Forms of reductive groups}
\label{section-forms-reductive}

\begin{lemma}
 \label{lemma-reductive-isomorphic-separable}
If $G, G'$ are two reductive groups over a field $k$ that are isomorphic over the algebraic closure $\bar k$, they are isomorphic over a finite separable extension $L\subset k^s$.
\end{lemma}

\begin{proof}
 This is a corollary of the existence of maximal $\bar k$-tori over $k$, Theorem \ref{algebraicgroups-theorem-maximal-tori-exist}, the fact that those split over a finite separable extension, Theorem \ref{algebraicgroups-theorem-diagonalizable-equivalence}, and the classification of split reductive groups in terms of root data, Theorem \ref{theorem-classification-with-universal-Cartan}.
\end{proof}

By ``continuous descent data'' over a separable closure $k^s$, in a $k$-groupoid $\mathcal F$ as above, we will mean pairs $(u,\phi)$ consisting of $u\in \mathcal F_{k^s}$ and isomorphisms $\phi_\gamma: \gamma u \to u$ for $\gamma\in\Gamma = \text{Gal}(k^s/k)$, such that \emph{these descent data are induced from descent data over a finite Galois extension $L/k$}, in the obvious way (i.e., $u$ is the base change of some object $u'\in \mathcal F_L$, and $\phi$ is obtained by extending scalars from some descent datum $\phi'$ over $L$).


\begin{proposition}
\label{proposition-forms-by-descent}
 The category of reductive groups over $k$ is equivalent to the category of continuous descent data $(G, \phi)$, where $G$ is a reductive group over a fixed separable closure $k^s$, and $\phi$ is an isomorphism $G\times_k k^s\simeq k^s\times_k G$. 
 
 Given a reductive group $G$ over $k$, the set of isomorphism classes of forms $G'$ of $G$ over $k$ is in natural bijection with the Galois cohomology set $H^1_{\text{cont}}(\Gamma, \text{Aut}(G_{k^s}))$, defined as the set of $G(k^s)$-orbits of \emph{continuous} $1$-cocycles $\Gamma\to \text{Aut}(G_{k^s})$ (i.e., factoring through a finite extension). 
\end{proposition}

\begin{proof}
 The first statement follows from Lemma \ref{lemma-reductive-isomorphic-separable}, and effectiveness of descent for linear algebraic groups (an easy corollary of Example \ref{example-descent-algebras}). The second follows from Proposition \ref{proposition-descent-cohomology}. 
\end{proof}

Thus, we are led to study the cohomology of the automorphism group of $G$ (over the separable closure). From now on, for every linear algebraic group over $k$, we will be writing simply $H^1(\Gamma, G)$ for the continuous cohomology set $H^1_{\text{cont}}(\Gamma, G(k^s))$. 

\begin{proposition}
 \label{proposition-inner-outer-cohomology}
Given a reductive group $G$ over a field $k$, there is an exact sequence
\begin{equation}
\label{equation-inner-outer-cohomology}
 H^1(\Gamma, \text{Inn}(G)) \to H^1(\Gamma, \text{Aut}(G)) \to H^1(\Gamma, \text{Out}(G)) \to 0.
\end{equation}
Moreover, if we assume that $G$ is split over $k$, and we fix a pinning $(G, B, T, \{p_\alpha\}_{\alpha\in\Delta})$ \emph{over $k$} (Definition \ref{definition-pinning}), the resulting splitting $\text{Out}(G)\to \text{Aut}(G)$ of Proposition \ref{proposition-automorphism-sequence} induces a splitting
$$  H^1(\Gamma, \text{Out}(G)) \to H^1(\Gamma, \text{Aut}(G))$$
whose image corresponds to the quasisplit forms of $G$. 

Moreover, the cohomology groups of \eqref{equation-inner-outer-cohomology}, together with this splitting, classify isomorphism classes in the following sequence of categories:
$$ \{\mbox{$G_{\text{ad}}$-torsors over $k$}\} \to \{\mbox{forms of $G$ over $k$}\} \rightleftarrows \{\mbox{quasisplit forms of $G$ over $k$}\},$$
compatibly with the inclusion functor on the right, and the functor that assigns to a right $G$-torsor $T$ the $G$-automorphism group $\text{Aut}^G(T)$.
\end{proposition}


\begin{proof}
 The sequence \eqref{equation-inner-outer-cohomology} follows immediately by applying the long exact sequence of cohomology to the short exact sequence of \eqref{equation-Inn-Aut-Out}. To prove surjectivity of the map to $H^1(\Gamma, \text{Out}(G))$, it is enough to assume that $G$ is split, since the choice of a different form only affects the base point. (Such a form always exists by Theorem \ref{theorem-classification-reductive}.) Then, the splitting $\text{Out}(G)\to \text{Aut}(G)$ of Proposition \ref{proposition-automorphism-sequence} gives rise to a splitting of the corresponding cohomology groups, and in particular proves surjectivity. Finally, the identification of cohomology groups with isomorphism classes of objects in the stated categories follows from effective descent and Proposition \ref{proposition-descent-cohomology}. In all cases, the stated groups are the groups of isomorphisms of a given object in the stated category (for example, $G_{\text{ad}}=\text{Inn}(G)$ is the automorphism group of a $G_{\text{ad}}$-torsor), except for the category of quasisplit forms of $G$ over $k$, which has more automorphisms than the outer automorphisms of $G$, and therefore requires some explanation. 
 
 By Proposition \ref{proposition-automorphism-sequence}, $\text{Out}(G)$ can be identified with automorphisms of the based root datum $\Psi^+(\mathbb A^G)$. By Theorem \ref{theorem-classification-with-pinning}, this can be identified with the automorphism group of a pinned quadruple $(G, B, T, p)$ (over $k^s$) or, equivalently, a quadruple $(G, B, T, \ell)$, where $\ell$ is an algebraic Whittaker datum for $B$ (Definition \ref{definition-pinning}). Such quadruples make sense over $k$, as well (and its Galois extensions), thus, by Galois descent, the set $ H^1(\Gamma, \text{Out}(G)) $ classifies isomorphism classes of such quadruples over $k$. But the forgetful map $(G, B, T, \ell)\to (G,B)$ is a bijection on isomorphism classes, because every pair $(G,B)$ admits a maximal torus $T\subset B$ and a Whittaker datum $\ell$, and any two such are conjugate by an element of $B_{\text{ad}}\subset G_{\text{ad}}$. [More details to be added.]
 
 
\end{proof}


\begin{definition}
 \label{definition-pure-inner-form}
A {\it pure inner form} of an algebraic group $G$ over $k$ is a $G$-torsor $T$; the term is often used to refer to the $G$-automorphism group of $T$, but with the understanding that a $G$-torsor has been fixed. An {\it inner form} of $G$ is a pure inner form $R$ for the adjoint group $G_{\text{ad}} = \text{Inn}(G)$; the term is often used to refer to the form $R\times^{\text{Inn}(G)} G$, but with the understanding that a $G_{\text{ad}}$-torsor has been fixed.
\end{definition}




\section{The $L$-group and the $C$-group}
\label{section-Lgroup}

Let $G$ be a reductive group over a field $k$, fix a separable closure $k^s$ of $k$ with Galois group $\Gamma$, and let $\Psi^+(G)$ be the associated based root datum of $G_{k^s}$, by Theorem \ref{theorem-classification-with-universal-Cartan}. Since the universal Cartan $\mathbb A^G$ is defined over $k$ (Proposition \ref{algebraicgroups-proposition-universal-Cartan}), and has a canonical set of positive roots, there is an action of the Galois group $\Gamma$ on its character group which preserves the positive roots, hence an action 
\begin{equation}
 \label{equation-Galoisaction-basedrootdatum}
\Gamma \to \text{Aut}(\Psi^+(G)).
\end{equation}

\begin{remark}
 \label{remark-Galoisaction-basedrootdatum}
By Proposition \ref{proposition-automorphism-sequence}, we have a canonical isomorphism $\text{Out}(G) \simeq \text{Aut}(\Psi^+(G))$. 
Viewed as a Galois cocycle into  $\text{Aut}(\Psi^+(G))$ (with trivial Galois action, corresponding to the split form of $G$), the corresponding cohomology class in $H^1(\Gamma, \text{Out}(G))$ is the one attached to the outer class of $G$ by Proposition \ref{proposition-inner-outer-cohomology}. 

Notice that the universal Cartan groups for inner forms are canonically isomorphic, in an order-preserving way, so that the action \eqref{equation-Galoisaction-basedrootdatum}, hence also the $L$-group that we are about to define, are identical for two groups that are inner forms of each other.
\end{remark}


\begin{definition}
 \label{definition-Lgroup}
Let $G$ be a reductive group over a field $k$, and fix a separable closure $k_s$ with Galois group $\Gamma$. Let $\Psi^+(G)$ be the based root datum of $G$, and $\check\Psi^+(G)$ the dual based root datum, obtained by interchanging the character and the cocharacter lattices. The {\it Langlands dual group} $\check G$ is the pinned group (provided by the equivalence of categories of Theorem \ref{theorem-classification-with-pinning}) with based root datum $\check\Psi^+(G)$. The {\it $L$-group} is the semidirect product ${^LG}=\check G \rtimes\Gamma$, where $\Gamma$ acts through \eqref{equation-Galoisaction-basedrootdatum}, by the pinned automorphisms provided by the equivalence of Theorem \ref{theorem-classification-with-pinning}. 
\end{definition}

Notice that the $L$-group can be thought of as being defined over any field, in fact over $\mathbb Z$. The ring of definition that one uses for the $L$-group depends on the coefficients of representations of $G$ that one considers.

\begin{remark}
 \label{remark-Lgroup-sheaf}
The choice of separable closure, and the definition of the $L$-group as a semidirect product (hence, a distinguished splitting $\Gamma \to {^LG}$) are not completely justified. It is better to think of the $L$-group as \emph{a sheaf of pinned reductive groups over the \'etale site of $\text{Spec} F$}. Hence, instead of talking about ``the'' based root datum of $G$ (defined using its weight lattice over a fixed separable extension), we have based root data for every separable extension over which $G$ splits, and isomorphisms between them for every morphism of such fields. In particular, the descent data give rise to the Galois action.
\end{remark}



Although the Langlands program with complex coefficients is often formulated in terms of the $L$-group, this is not the most natural dual group to consider, and in particular does not work well with integer coefficients. A more natural choice is the \emph{$C$-group}, which we introduce now, following \cite{Buzzard-Gee} and ideas of Joseph Bernstein. The description of \cite{Buzzard-Gee} is combinatorial, but, following Bernstein, we will describe the $C$-group in terms of the Langlands dual of a natural extension of $G$.

\begin{lemma}
 \label{lemma-squareroot-canonicalbundle}
Let $G$ be a connected reductive group, $\mathcal B$ its flag variety of Borel subgroups, $\Omega$ the canonical bundle (bundle of volume forms) on $\mathcal B$. Then, there is a canonical square root $\Omega^\frac{1}{2}$, i.e., a line bundle over $\mathcal B$ whose square is $\Omega$, characterized by the fact that it admits a linearization for the action of the simply connected cover $G_{sc}$ of the derived group of $G$. 
\end{lemma}

We recall that ``linearization'' means an action of the group on the bundle, i.e., on its sheaf of sections, compatible with its action on the base, i.e., an isomorphism of the two pullbacks under the projection and action maps $G_{sc} \times \mathcal B \rightrightarrows \mathcal B$ which is compatible with the group structure.

\begin{proof}
By Galois descent and the uniqueness of the square root with this property, it is enough to assume that the field is algebraically closed. If $B\in \mathcal B$ is a Borel subgroup, $2\rho$ is the sum of positive roots of $G$, and we use exponential notation when thinking about the corresponding character $e^{2\rho}:B\to \mathbb G_m$, then $\Omega$ is a $G$-linear bundle under the right action of $G$ on $\mathcal B$, and $B$ acts by the character $e^{2\rho}$ on its fiber [exercise!]. In other words, $\Omega$ is induced from the character $e^{2\rho}$ of $B$; its total space is $\mathbb Ga_{2\rho} \times^B G$,
where $\mathbb Ga_{2\rho}$ is the line with an action of $B$ by this character.

If $B_{sc}$ is the corresponding Borel of $G_{sc}$, then $\mathcal B = B\backslash G=B_{sc}\backslash G_{sc}$, and the restriction of $2\rho$ to $B_{sc}$ admits a unique square root, namely $\rho$. Thus, there is a unique $G_{sc}$-linear bundle $\Omega^{\frac{1}{2}}$ over $\mathcal B$ whose square is $\Omega$.
\end{proof}



\begin{definition}
 \label{definition-canonical-extension}
Let $G$ be a connected reductive group, $\mathcal B$ its flag variety of Borel subgroups, $\Omega^\frac{1}{2}$ the canonical square root of the canonical bundle on $\mathcal B$ (Lemma \ref{lemma-squareroot-canonicalbundle}). The {\it canonical extension} $\tilde G$ is the group of pairs $(g,\tau)$, where $g\in G$ and $\tau$ is an isomorphism: $g^*\Omega^\frac{1}{2}\xrightarrow\sim \Omega^\frac{1}{2}$.
\end{definition}

\begin{remark}
 \label{remark-canonical-extension}
Notice that we consider $G$ as acting on the right on $\mathcal B$, so pullback is a left action on line bundles: $(g_1g_2)^*\Omega = g_1^*(g_2^*\Omega)$. The composition law on the extended group is $(g_1,\tau_1)\cdot (g_2,\tau_2) = (g_1g_2, \tau_1\circ(g_1^*\tau_2))$. The extended group in a central linear algebraic extension:
\begin{equation}
\label{equation-canonical-extension} 1\to \mathbb G_m \to \tilde G \to G \to 1,
\end{equation}
where the central $\mathbb G_m$ is the group of scalar automorphisms of the line bundle $\Omega^\frac{1}{2}$. It acts on $\mathcal B$ through the quotient $G$, and, by definition, \emph{the line bundle $\Omega^\frac{1}{2}$ is canonically $\tilde G$-linear}.

On the other hand, if $G$ is simply connected, then, by the definition of $\Omega^\frac{1}{2}$, it already has a (unique) $G$-linearization; hence, in this case, $\tilde G = \mathbb G_m \times G$, canonically.
\end{remark}


\begin{definition}
 \label{definition-C-group}
For a reductive group $G$ over a field $k$ (in a fixed separable closure with Galois group $\Gamma$), the {\it extended Langlands dual group} $\check{\tilde G}$ of $G$ is the Langlands dual group (Definition \ref{definition-Lgroup}) of the canonical extension $\tilde G$ (Definition \ref{definition-canonical-extension}), and the {\it $C$-group} ${^CG}$ is the $L$-group of the canonical extension. 

The $C$-group comes with a canonical character dual to the sequence \eqref{equation-canonical-extension}:
\begin{equation}
\label{equation-cyclotomic-character} 
1\to {^LG} \to {^CG} \to \mathbb G_m \to 1,
\end{equation}
which will be called the {\it cyclotomic character} of the $C$-group.
\end{definition}

\begin{remark}
 \label{remark-Cgroup-rootdata}
In terms of based root data, if $X$ denotes the weight lattice in $\Psi^+(G)$, and $\check X$ the coweight lattice, the weight lattice $\tilde X$ of $\tilde G$ is generated inside of the vector space $X_{\mathbb Q} \oplus \mathbb Q$ by $X$ and the element $\tilde\rho = (\rho, 1)$; its projection $\mathbb Z\subset \mathbb Q$ is the weight lattice of the central $\mathbb G_m$, and $\tilde\rho$ is the character from which the $\tilde G$-linear line bundle $\Omega^\frac{1}{2}$ is induced. 

Notice that the element $e^{2\rho}(-1)$ is a canonical central element of the dual group $\check G$ (possibly trivial), thus defining a map $\mu_2\to \check G$. The extended dual group $\check{\tilde G}$ is the group $\check G \times^{\mu_2} \mathbb G_m$. 
\end{remark}





\section{The real case: compact Lie groups and Lie algebras}
\label{section-real-compact}

The main reference for this section is \cite[Ch.\ IX \S 1]{Bourbaki-Lie}.

\begin{definition}
 \label{definition-Inng}
Let $\mathfrak g$ be a finite-dimensional real or complex Lie algebra. The {\it group of inner automorphisms} $\text{Inn}(\mathfrak g)$ is the connected immersed subgroup of $\text{GL}(\mathfrak g)$ (see \ref{liegroups-general-proposition-Liesubalg}) whose Lie algebra is $\text{ad}(\mathfrak g)$.
\end{definition}

For compact or semisimple Lie groups, this coincides with the group $\mathcal E(\mathfrak g)$ of Definition \ref{liestructure-definition-Eg}, but we will not prove that.



\begin{proposition}
 \label{proposition-compact-Lie}
Let $\mathfrak g$ be a (finite-dimensional) real Lie algebra. The following conditions are equivalent:
\begin{enumerate}
 \item $\mathfrak g$ is isomorphic to the Lie algebra of a compact Lie group.
 \item The group $\text{Inn}(\mathfrak g)$ is compact.
 \item $\mathfrak g$ has a positive definite, invariant symmetric bilinear form.
 \item The adjoint representation of $\mathfrak g$ is semisimple, and for every $x\in \mathfrak g$, $\text{ad}(x)$ is semisimple with purely imaginary eigenvalues.
 \item $\mathfrak g$ is a direct sum of its center and its maximal semisimple ideal, and the Killing form is negative semi-definite. 
\end{enumerate}

If, moreover, $\mathfrak g = \text{Lie}(G)$ for some connected Lie group $G$, the above are equivalent to any of the following:
\begin{enumerate}\setcounter{enumi}{5}
 \item The group $\text{Ad}(G)\subset \text{GL}(\mathfrak g)$ is compact.
 \item There is a Riemannian metric on $G$ invariant under left and right translations.
\end{enumerate}

Finally, in that case the exponential map $\mathfrak g \to G$ is surjective.
\end{proposition}

\begin{proof}
 $(1) \Rightarrow (2)$: If $\mathfrak g = \text{Lie}(G)$ with $G$ compact and connected, then $\text{Inn}(\mathfrak g) = G/Z$, where $Z$ is the center, hence is compact. 
 
 $(2) \Rightarrow (3)$: Choose any positive definite symmetric bilinear form on $\mathfrak g$, and average over the compact group $\text{Inn}(G)$ to obtain an invariant one.
 
 $(3) \Rightarrow (4)$: The orthogonal complement, with respect to a definite invariant form, of an invariant subspace is invariant, which proves semisimplicity. The operators $\text{ad}(x)$ are then anti-self-adjoint, which implies that they are diagonalizable with purely imaginary eigenvalues.
 
 $(4) \Rightarrow (5)$: If the adjoint representation is semisimple, $\mathfrak g$ is the direct sum of its center and its simple ideals. If $B$ denotes the Killing form, then $B(x,x) = \text{tr}(\text{ad}(x)^2)$, and since the eigenvalues are all imaginary, this is $\le 0$. 
 
 $(5) \Rightarrow (1)$: The center of the Lie algebra is isomorphic to the Lie algebra of a compact real torus, so we are reduced to the case that $\mathfrak g$ is semisimple, with negative Killing form. Then, $\text{Inn}(\mathfrak g)$ is a closed [reference] subgroup of $\text{GL}(\mathfrak g)$, and since it preserves the negative definite form $B$, it belongs to the compact orthogonal subgroup $O(B)$, thus is compact (with Lie algebra $\mathfrak g$). 
 
 If $\mathfrak g = \text{Lie}(G)$, with $G$ connected, then $\text{Ad}(G)=\text{Inn}(\mathfrak g)$, as one can confirm by comparing Lie algebras. The last claim follows from the rest by constructing an $\text{Ad}(G)$-invariant positive definite form on $\mathfrak g$ (by the same averaging argument as above), and translating it by the left (or right) $G$-action. Vice versa, any invariant Riemannian metric, restricted to the tangent space at the identity, is such a form.
 
 Finally, one can check [Exercise!] that, for a left- and right-invariant Riemannian metric, the group-theoretic exponential map coincides with the Riemannian exponential map, hence $G$ is geodesically complete, and any two points can be joined by a length-minimizing geodesic (the \emph{Hopf--Rinow theorem}). 
\end{proof}

\begin{definition}
 \label{definition-compact-Lie}
A real Lie algebra $\mathfrak g$ satisfying the equivalent conditions of Proposition \ref{proposition-compact-Lie} is called a {\it compact Lie algebra}. 
\end{definition}

\begin{theorem}
 \label{theorem-compact-Lie}
The Lie algebra of a connected Lie group $G$ is compact if and only if $G$ is the surjective image of a morphism $V\times T\times K\to G$ with finite kernel, where $V$ is a vector space, $T$ is a compact real torus, and $K$ is a compact semisimple Lie group. 
\end{theorem}

\begin{proof}
 The direction $\Leftarrow$ follows from Proposition \ref{proposition-compact-Lie}.

 As seen in Proposition \ref{proposition-compact-Lie}, a compact Lie algebra is the direct sum of its center and a semisimple compact Lie algebra, and the connected component of the center has to be equal to a group of the form $V\times T$ as above, therefore we are reduced to the case when $\mathfrak g$ is compact and semisimple.
 
 In that case, the center $Z$ is discrete, and $G/Z=\text{Ad}(G)$ is compact by Proposition \ref{proposition-compact-Lie}, hence it suffices to prove that the group $G_{\text{ad}}=G/Z$ cannot be the image of a central isogeny with infinite kernel from a connected Lie group. Let $H\to G_{\text{ad}}$ be a central isogeny of finite degree. By Theorem \ref{liegroups-general-theorem-Weyl}, $H$ and $G_{ad}$ coincide with the real points of algebraic groups over $\mathbb R$, and the morphism between them is algebraic. Thus, we can apply the analysis of central isogenies of Theorem \ref{theorem-classification-reductive}, and deduce that, since $\mathfrak g$ is semisimple, the degree of any central isogeny is bounded by the index of the coroot lattice in the cocharacter lattice. Thus, $G$ is compact.
\end{proof}

\begin{proposition}
 \label{proposition-unique-split-compact-form}
Any semisimple Lie algebra $\mathfrak g$ over $\mathbb C$ has a unique split real form up to ($\text{Inn}(\mathfrak g)$-)conjugacy, and a unique compact form up to conjugacy. 
\end{proposition}

\begin{proof}
 Let $\mathfrak h \subset \mathfrak g$ be a Cartan subalgebra, and let $\mathfrak h_0$ be the real subspace spanned by the images of $\mathbb R$ under the cocharacters $\check\Phi\ni \check\alpha: \mathbf{G}_a\to \mathfrak h$. The Chevalley construction (omitted) shows that it extends to a real form $\mathfrak g_0$ of $\mathfrak g$, such that, for every root $\alpha$, $\mathfrak g_{\alpha,0}:= \mathfrak g_0 \cap \mathfrak g_\alpha$ is a real form of the root space $\mathfrak g_\alpha$. Those real forms, for $\alpha$ in a basis $\Delta$ of the root system, determine the form, and in turn are determined by a pinning (Definition \ref{definition-pinning}) relative to $\Delta$. All such pinnings are conjugate under the image of $\text{exp}(\mathfrak h)$ in $\text{Inn}(\mathfrak g)$, hence by the conjugacy of Cartan subalgebras we conclude that all split forms are conjugate. 

 Starting now with such a split form, and a pinning $\mathfrak g_\alpha\simeq\mathbf G_a$ for $\alpha\in\Delta$, let $X_\alpha \in \mathfrak g_\alpha$ be the element corresponding to $1\in\mathbf G_a$, and let $X_{-\alpha}\in \mathfrak g_{-\alpha}$ be the element forming an $\mathfrak{sl}_2$-triple $(h_\alpha, X_\alpha, X_{-\alpha})$, with $h_\alpha\in\mathfrak h$ the coroot. We can similarly choose elements $X_\alpha\in\mathfrak g_\alpha$ for all roots, such that $[X_\alpha, X_\beta] = N_{\alpha, \beta} X_{\alpha+\beta}$ whenever $\alpha+\beta$ is also a root, with constants $N_{\alpha,\beta}$ satisfying $N_{\alpha,\beta} = N_{\beta,\alpha}$. (This is part of the Chevalley construction, see \cite[Ch.\ VIII \S 2]{Bourbaki-Lie}.) Then, we define another real form of $\mathfrak g$ by 
 $$ \mathfrak g_c = i \mathfrak h_0 \oplus \bigoplus_{\{\pm \alpha\}\in \Phi/\{\pm 1\}} \left( \mathbb R(X_\alpha+X_{-\alpha}) \oplus  i \mathbb R(X_\alpha - X_{-\alpha})\right ).$$
 
 One checks that this is a compact form, \cite[Ch.\ IX \S 3.2]{Bourbaki-Lie}.  Obviously, this depends only on the pinning, not on the choices of $X_\alpha$ for $\alpha$ not simple, and is generated by the summands corresponding to simple roots $\alpha$. 

  
 For any other compact form $\mathfrak g_c'$, we will show that it is of this form; by the conjugacy of pinnings, we will then conclude that $\mathfrak g_c'$ is $G$-conjugate to $\mathfrak g_c$. 
 
 To lighten notation we may already use the conjugacy of Cartan subalgebras and assume that $\mathfrak g_c'$ contains a form $\mathfrak h_c'$ of $\mathfrak h$. By one of the equivalent conditions of Proposition \ref{proposition-compact-Lie}, the eigenvalues for the adjoint action of $\mathfrak h_c'$ on $\mathfrak g$ have to be purely imaginary, so we must have $\mathfrak h_c' = i \mathfrak h_0$. Consider the semilinear automorphism $\tau:\mathfrak g\to\mathfrak g$ corresponding to complex conjugation with respect to $\mathfrak g_c'$. It acts by $-1$ on $\mathfrak h_0$, hence interchanges the spaces $\mathfrak g_\alpha$ and $\mathfrak g_{-\alpha}$. If, for a simple $\alpha\in \Delta$ and $X_\alpha$, $X_{-\alpha}$ as above, we have $\tau (X_\alpha) = c X_{-\alpha}$, then we claim that $c>0$. 
 
 Indeed, if $B$ is the Killing form, $0 > B(X_\alpha, X_{-\alpha}) = c^{-1} B(X_\alpha, \tau(X_{-\alpha}))$, and the quadratic form $X\mapsto B(X,\tau(X))$ is negative definite on $\text{Res}_{\mathbb C/\mathbb R} \mathfrak g$, since $B$ is a $\mathbb C$-linear quadratic form on $\mathfrak g$ that is negative on the fixed space $\mathfrak g_c'$ of the involution $\tau$. 
 
 If we set $X_\alpha' = c^{-\frac{1}{2}} X_\alpha$, $X_{-\alpha}' = c^{\frac{1}{2}} X_\alpha'$, we obtain another $\mathfrak{sl}_2$-triple $(h_\alpha, X_\alpha', X_{-\alpha}')$ with $\tau(X_\alpha) = X_{-\alpha}$.  Then, the form $\mathfrak g_c'$ is generated by the elements $X_\alpha'+X_{-\alpha}'$ and $i(X_\alpha' -X_{-\alpha}')$ with $\alpha\in \Delta$, hence is of the stated form. 
\end{proof}


We deduce that any complex, connected semisimple group has a unique compact form up to conjugacy, and, in fact, arrive at the following strengthening of Theorem \ref{liegroups-general-theorem-Weyl}, whose formulation is taken from class notes of Brian Conrad:


\begin{theorem}
\label{theorem-functor-compactgroups}
 The functor $G\mapsto G(\mathbb R)$ is an equivalence between: the category of $\mathbb R$-anisotropic reductive $\mathbb R$-groups whose connected components have $\mathbb R$-points, and the category of compact Lie groups. If $G$ is such an $\mathbb R$-group then $G^0(\mathbb R)=G(\mathbb R)^0$. The $\mathbb R$-group $G$ is semisimple if and only if $G(\mathbb R)$ has finite center, and in such cases $G^0$ is simply connected in the sense of algebraic groups if and only if $G(\mathbb R)^0$ is simply connected in the sense of topology.
\end{theorem}


\begin{proof}

 [To be added]
\end{proof}



\section{Classification of real forms in terms of Cartan involutions}
\label{section-real-Cartan}

Let $G$ be a connected, complex reductive group. Proposition \ref{proposition-unique-split-compact-form} implies that it has a unique, up to conjugacy, split real form. [Details omitted for now --- need to explain lift from Lie algebra to group.] Fixing that form, by Proposition \ref{proposition-inner-outer-cohomology} we obtain a bijection between isomorphism classes of real forms of $G$, and $H^1(\Gamma, \text{Aut}(G))$, where $\Gamma \simeq \mathbb Z/2$. Equivalently, forms correspond bijectively to sections
$$ \mathbb Z/2 \to \text{Aut}(G) \rtimes \mathbb Z/2$$
up to $\text{Aut}(G)$-conjugacy. 

Here, we will discuss a more classical (and useful) description of real forms, in terms of actual homomorphisms 
$$ \mathbb Z/2 \to \text{Aut}(G)$$
up to $\text{Aut}(G)$-conjugacy. This gives rise to an extremely important correspondence between real forms of $G$ and \emph{symmetric spaces}, that is, spaces of the form $G/G^\theta$, where $\theta$ is an involution (automorphism of order two) of $G$. 




\begin{definition}
 \label{definition-Cartan-decomposition-involution} 
If $\mathfrak g_0$ is a semisimple Lie algebra over $\mathbb R$, with complexification $\mathfrak g$ and associated antiholomorphic involution $\sigma$, a {\it Cartan involution} for $\mathfrak g_0$ is a holomorphic involution $\theta$, which commutes with $\sigma$, such that $\mathfrak g^{\theta\sigma}$ is a compact form of $\mathfrak g$ (Definition \ref{definition-compact-Lie}). 
 
Given such a Cartan involution, the associated {\it Cartan decomposition} is the pair $(\mathfrak k,\mathfrak p)$ of complementary vector subspaces of $\mathfrak g_0$, where $\mathfrak k = \mathfrak g_0^\theta$ (in particular, $\mathfrak k$ is a compact Lie subalgebra) and $\mathfrak p = \mathfrak g_0^{-\theta}$. Since the pair $(\mathfrak k,\mathfrak p)$ obviously determines the involution $\theta$, we will say that $(\mathfrak k,\mathfrak p)$ is a Cartan decomposition, without reference to $\theta$. 
 
A {\it Cartan decomposition} for a real Lie group $G$ is a pair $(K,\mathfrak p)$, where $K\subset G$ is a compact subgroup, and $\mathfrak p\subset\mathfrak g$ is an $\text{Ad}(K)$-stable subspace such that the map $K\times \mathfrak p \ni (k,X)\mapsto k\exp(X)$ is a diffeomorphism.

Given such a Cartan decomposition, the map $\theta(k \exp(X)) = k \exp(-X)$ (where $k\in K$, $X\in\mathfrak p$) is an involution of $G$, called the {\it Cartan involution}. Since $\theta$ determines $K = G^\theta$ and $\mathfrak p = \mathfrak g_0^{-\theta}$ uniquely, we will refer to such a $\theta$ as a Cartan involution, without reference to $(K,\mathfrak p)$.
\end{definition}


\begin{example}
 \label{example-Cartan-decomposition-GL}
Let $G=\text{GL}_n$, $K = O(n)$, the compact orthogonal group of the standard inner product on $\mathbb R^n$, $\mathfrak k = \text{Lie}(K)$ and $\mathfrak p=$ the subspace of symmetric matrices in $\mathfrak g_0 = \text{Lie}(G) = \mathfrak{gl}(n, \mathbb R)$. 

Then, the pair $(\mathfrak k, \mathfrak p)$ is a Cartan decomposition of $\mathfrak g_0$, and the pair $(K,\mathfrak p)$ is a Cartan decomposition of $G$. Indeed, the quotient space $G/K$ can be identified with the space of positive definite quadratic forms on $\mathbb R^n$, each represented by a symmetric matrix $A$ with positive eigenvalues, and each such matrix $A$ has a unique square root $B$ of the same type, which in turn is the exponential of a unique symmetric matrix $X$, so $A = B^2 = \exp(X)^2 = \exp(X) \exp(X)^t$, uniquely. 


The Cartan involution on $G$ is $g\mapsto g^{-t}$, on $\mathfrak g$ it is $X\mapsto - X^t$, and the associated compact form of $\mathfrak g = \mathfrak{gl}(n, \mathbb C)$ is the Lie algebra of the unitary group for the standard inner product on $\mathbb C^n$.
\end{example}



\begin{proposition}
 \label{proposition-Cartan-decomposition}
If $G$ is a connected semisimple Lie group (i.e., with semisimple Lie algebra) with Lie algebra $\mathfrak g_0$, $(\mathfrak k,\mathfrak p)$ is a Cartan decomposition of $\mathfrak g_0$, and $K = \exp(\mathfrak k)$. If $G$ has finite center, then $K$ is a compact subgroup, and $(K, \mathfrak p)$ is a Cartan decomposition of $G$.
\end{proposition}

\begin{proof}
 The proof reduces to the analogous statement for $G=\text{GL}_n(\mathbb R)$, $K=O(n)$, as follows: 
 
 First of all, let $K'$ be the immersed Lie subgroup of $G$ with Lie algebra $\mathfrak k$, see Proposition \ref{liegroups-general-proposition-Liesubalg}. Since $\mathfrak k$ is compact, by Proposition \ref{proposition-compact-Lie}, the exponential map $\mathfrak k \to K'$ is surjective, hence $K'=K$. 
 
Let $\theta$ be the associated Cartan involution, let $B$ be the Killing form, and consider the quadratic form $q:X\mapsto B(X,\theta X)$, which is negative definite on $\mathfrak g_0$. One easily checks that, with respect to this form, $\text{ad}(X)$ is
\begin{itemize}
 \item symmetric, if $X\in \mathfrak p$;
 \item skew-symmetric, if $X\in \mathfrak k$. 
\end{itemize}

Thus, $\text{Ad}_{\mathfrak g}(K)\subset O(q)\subset \text{GL}(\mathfrak g)$, and by Example \ref{example-Cartan-decomposition-GL} the map $\text{Ad}_{\mathfrak g}(K) \times \mathfrak p \ni (g, X) \mapsto g \exp(\text{ad}(X)) \in \text{GL}(\mathfrak g)$ is a closed embedding. (Notice that $\text{ad}$ has no kernel, by semisimplicity.) Its image is both open and closed in $\text{Ad}(G)$, and by connectedness it is equal to $\text{Ad}(G)$. Therefore, the map $K\times \mathfrak p \to G$ is an open and closed embedding, and again by connectedness it is an isomorphism. Note that $K$ is compact, because by assumption the kernel of $K\to \text{Ad}_{\mathfrak g}(K)$ is finite.

\end{proof}



We will eventually see that any reductive Lie group admits a Cartan decomposition, but for now we restrict our attention to complex groups (considered, by restriction of scalars, as real Lie groups). 

\begin{proposition}
 \label{proposition-Cartan-decomposition-complex}
Let $G$ be a complex, connected semisimple algebraic group, and $\mathfrak k\subset \mathfrak g$ a compact form of $\mathfrak g$, which exists, by Proposition \ref{proposition-unique-split-compact-form}, uniquely up to conjugacy. Let $\mathfrak p = i \mathfrak k \subset \mathfrak g$, and set $K = \exp(\mathfrak k)$. Then, $(K, \mathfrak p)$ is a Cartan decomposition of $G(\mathbb C)$ (viewed as a real Lie group), in the sense of Definition \ref{definition-Cartan-decomposition-involution}. Moreover, the normalizer of $K$ is $K\exp(i \mathcal Z(\mathfrak k))$, where $\mathcal Z(\mathfrak k)$ denotes the center of $\mathfrak k$. 
\end{proposition}


\begin{proof}
 Indeed, one immediately checks that, since the Killing form $B_{\mathfrak g}$ is negative definite on the real form $\mathfrak k$ of $\mathfrak g$, the Killing form $B_{\mathfrak g \otimes_{\mathbb R} \mathbb C}$ is negative definite on the real form $\mathfrak k \oplus i \mathfrak p$ of $\mathfrak g \otimes_{\mathbb R} \mathbb C$, hence this is a compact form. The first claim now follows from Proposition \ref{proposition-Cartan-decomposition}, taking into account that complex semisimple groups have finite fundamental groups, and therefore finite center (see the argument in the proof of Theorem \ref{theorem-compact-Lie}).
 
 For the normalizer $\mathcal N_G(K)$ of $K$, it suffices to show that $\mathcal N_G(K) \cap \exp(\mathfrak p) = \exp(i \mathfrak Z(\mathfrak k))$. If $X\in \mathfrak p$ is such that $\exp(X)$ normalizes $K$, then $[X,\mathfrak k]\subset \mathfrak k$, but on the other hand $[\mathfrak p, \mathfrak k]\subset \mathfrak p$, so $[X,\mathfrak k]=0$, i.e., $X \in \mathfrak p^{\mathfrak k} = (i\mathfrak k )^{\mathfrak k} = i \mathcal Z(\mathfrak k)$. 
\end{proof}



Now we have the following, see also \cite{Adams-Taibi}:

\begin{proposition}
 \label{proposition-holomorphic-antiholomorphic}
Let $G$ be a connected complex semisimple algebraic group. Then:

\begin{enumerate}
 \item Any holomorphic, resp.\ antiholomorphic, involution of $G(\mathbb C)$ is algebraic, i.e., induced by an algebraic involution of $G$, resp.\ a conjugate-linear involution of $\text{Res}_{\mathbb C/\mathbb R} G$.
 
 For the rest of the statements, write $G$ for $G(\mathbb C)$.

 \item If $G$ is abelian, it admits a unique compact antiholomorphic involution $\tau$ (i.e., such that $G^\tau$ is compact). It commutes with all holomorphic or antiholomorphic involutions of $G$.

 \item For any antiholomorphic involution $\sigma$ of $G$ there exists a holomorphic involution $\theta$, commuting with $\theta$, such that $G^{\theta\sigma}$ is compact, and vice versa: for every such $\sigma$ there exists such a $\theta$. 

Moreover, $\theta$ is unique up to $(G^\sigma)^0$-conjugacy, and $\sigma$ is unique up to $(G^\theta)^0$-conjugacy. 
\end{enumerate}
\end{proposition}


\begin{proof}

For the first statement, see \cite[Lemma 3.1]{Adams-Taibi}.

If $G$ is abelian, it is a torus, and it is easy to see that the only compact antiholomorphic involution $\tau$ is the one which, on $\mathfrak g$, acts by $-1$ on the real subspace $E$ generated by the differentials of cocharacters, and by $+1$ on $iE$. Any other holomorphic or antiholomorphic involution has to preserve the cocharacter lattice, hence the eigenspaces $E$ and $iE$ for $\tau$, hence commutes with $\tau$. 

Any involution of $G$ induces involutions on its derived group $G_{\text{der}}$ and its center $\mathcal Z(G)$, and since $G = \mathcal Z(G) G_{\text{der}}$, the rest of the statements are now reduced to $G_{\text{der}}$. Thus, we may assume that $G$ is semisimple.

By Proposition \ref{proposition-unique-split-compact-form}, the set $M$ of compact forms of $\mathfrak g$ is a homogeneous space under $G$, and if we fix a compact form $G_c$, it can be identified with the quotient $G/G_c \exp(i \mathcal Z(\mathfrak g_c))$. The tangent space at the point $x=G_c$ can be identified with the quotient $\mathfrak p_{\text{ad}} = \mathfrak p/i \mathcal Z(\mathfrak g_c)$ of $\mathfrak p = i \mathfrak g_c$, and the Cartan decomposition, Proposition \ref{proposition-Cartan-decomposition}, shows that there is a well-defined exponential map 
$$\exp_x: \mathfrak p_{\text{ad}}  = T_x M \to M,$$
descending from the exponential map on $\mathfrak p$, which is an isomorphism. 

Given $\sigma$, it induces an automorphism of order $2$ of $M$, and our goal is to find a fixed point. Start with any point $x = G_c$, then $\sigma$ induces an isomorphism $\sigma_*$ between $T_x M$ and $T_{\sigma(x)} M$. We claim:

\begin{quote}
 $\sigma$ commutes with the exponential map, i.e., $\exp_{\sigma(x)} (\sigma_* X) = \sigma(\exp_x(X))$, for every $X \in T_x M$. 
\end{quote}

Indeed, since $\sigma$ is antiholomorphic, it maps $\mathfrak p = i \mathfrak g_c$ to $i\sigma(\mathfrak g_c)$, which is the space analogous to $\mathfrak p$ for $\sigma(x)$.

Now, let $X\in T_x M$ be the unique element with $\sigma(x) = \text{exp}_x(X)$, choose a preimage $\tilde X$ of $X$ in $\mathfrak p$, and set $g = \exp(\tilde X)$ (the exponential in the group, here). Hence, $\sigma(x) = g x$, and translation by $g$ identifies the tangent space of $x$ with that of $\sigma(x)$, by a map which we will denote by $g_*$ (and its inverse by $g^*$). 

We claim that $g^*\sigma_*(X) = -X$. Indeed, $x = \sigma(\sigma(x)) =\sigma (\exp_x (X)) = \exp_{\sigma(x)} (\sigma_* X)$ by the claim above, and on the other hand $x = g^{-1}(\sigma(x)) = \exp(-\tilde X) (\sigma (x)) = \exp_{\sigma(x)} (g_*(-X))$, and by the fact that the map $\exp_{\sigma(x)}$ is an isomorphism, we deduce that $\sigma_* X = g_*(-X)$, or equivalently $g^*\sigma_* X = -X$.

Hence, the point $\exp_x (\frac{X}{2}) = \exp(\frac{\tilde X}{2}) x$ is fixed under $\sigma$, because 
$$\sigma (\exp_x (\frac{X}{2})) = \exp_{\sigma(x)} (\sigma_* \frac{X}{2}) = \exp_{\sigma(x)} (- g_* \frac{X}{2})=$$
$$= \exp(-\frac{\tilde X}{2}) \sigma (x) = \exp(-\frac{\tilde X}{2}) \exp(\tilde X) x = \exp(\frac{\tilde X}{2}) x.$$ 

The proof for $\theta$ in place of $\sigma$ is identical. 


Now we show that all points of $M^\sigma$ lie in the same $(G^\sigma)^0$-orbit. Given two such points $x, x'$, we can write $x' = \exp_x(X)$, for a unique element $X\in T_x M$. It is enough to show that $X \in (T_x M)^\sigma$, because then (by semisimplicity) it can be lifted to $\tilde X \in \mathfrak p^\sigma$, and then $\exp(\tilde X)$ will belong to $(G^\sigma)^0$. Again, by the $\sigma$-equivariance of the exponential map, we get that $\exp_x(X) = x' = \sigma(x') = \sigma(\exp_x(X)) = \exp_x (\sigma_* X)$, and by the fact that $\exp_x$ is an isomorphism we deduce that $\sigma_*X = X$.

The proof or $\theta$ is, again, identical.

\end{proof}

This leads to the following two theorems which are the main results of this subsection:

\begin{theorem}
 \label{theorem-Cartan-involution-exists}
(The group of $\mathbb R$-points of) any connected reductive algebraic group $G$ over $\mathbb R$ admits a Cartan decomposition $(K,\mathfrak p)$, see Definition \ref{definition-Cartan-decomposition-involution}. The group $K$ is a maximal compact subgroup of $G=G(\mathbb R)$, and all maximal compact subgroups (and all Cartan decompositions) are $G(\mathbb R)^0$-conjugate. 
\end{theorem}

\begin{proof}
 If $\sigma$ denotes the antiholomorphic involution of $G(\mathbb C)$ which fixes $G(\mathbb R)$, by  Proposition \ref{proposition-holomorphic-antiholomorphic} there exists a commuting holomorphic involution $\theta$ such that $\sigma\theta$ is compact, unique up to $G(\mathbb R)^0$-conjugacy. The restriction of $\theta$ to $G(\mathbb R)$ induces a Cartan decomposition $(K, \mathfrak p)$ with $K = G(\mathbb R)^\theta$ and $\mathfrak p = \text{Lie}(G(\mathbb R))^{-\theta}$. The group $K$ is maximal compact, because if $K'\sup K$ were compact, and $g \in K'$ did not belong to $K$, then by the Cartan decomposition $g = k\exp(X)$ for some $X\ne 0$ in $\mathfrak p$, but then $\exp(X)$ belongs to $K'$, but the powers of $\exp(X)$ have no accumulation point, a contradiction.
\end{proof}

\begin{remark}
 \label{remark-no-Cartan}
Not every non-algebraic semisimple connected Lie group (i.e., with semisimple Lie algebra) admits a Cartan decomposition. For example, by the Cartan decomposition, $K=S^1$ is a deformation retract of $G=\text{SL}_2(\mathbb R)$, hence $\pi_1(\text{SL}_2) \simeq \mathbb Z$, and its fundamental cover $\widetilde{\text{SL}_2}$ is a semisimple Lie group with infinite center, no nontrivial compact subgroups (because the preimage of $K$ is isomorphic to $\mathbb R$), and since the exponential map is not bijective, it does not admit a Cartan decomposition. 
\end{remark}



\begin{theorem}
\label{theorem-holomorphic-antiholomorphic}
Let $G$ be a connected complex reductive group. Let $Ant$ denote the space of antiholomorphic involutions on $G$, and $Hol$ the space of holomorphic involutions. There is a canonical bijection 
$$Ant/G \leftrightarrow Hol/G$$
induced by the distinguished $G$-orbit on $Ant \times Hol$ of those pairs $(\sigma,\theta)$ such that $\sigma$ commutes with $\theta$, and $\mathfrak g^{\sigma\theta}$ is compact.

If $\Gamma =\text{Gal}(\mathbb C/\mathbb R)$, and we fix a compact real form (an antiholomorphic involution) $\tau$ on $G$, this bijection gives rise to bijections of pointed sets
$$H^1(\Gamma, \text{Aut}(G)) \xrightarrow\sim H^1(\mathbb Z/2, \text{Aut}(G)),$$
where $\mathbb Z/2$ acts trivially on $\text{Aut}(G)$, i.e., \\ $H^1(\mathbb Z/2, \text{Aut}(G)) = \Hom(\mathbb Z/2, \text{Aut}(G))/\text{Aut}(G)-\text{conj}$, as well as
$$Z^1(\Gamma, \text{Aut}(G))/\text{Inn}(G) \xrightarrow\sim \Hom(\mathbb Z/2, \text{Aut}(G))/\text{Inn}(G),$$
where $Z^1$ denotes the set of $1$-cocycles.
\end{theorem}

\begin{proof}
 Indeed, Proposition \ref{proposition-holomorphic-antiholomorphic} states that the set of those pairs $(\sigma,\theta)$ such that $\sigma$ commutes with $\theta$, and $\mathfrak g^{\sigma\theta}$ is compact, forms a unique $G$-orbit which surjects onto both $Ant$ and $Hol$. 
 
 If $\tau$ is the antiholomorphic involution corresponding to a compact form, used to define the Galois action on $\text{Aut}(G)$ by pre- and post-composition, for any $\kappa \in Z^1(\Gamma, \text{Aut}(G))$, identified with the image in $\text{Aut}(G)$ of complex conjugation, the composition $\sigma = \tau\kappa$ is also an antiholomorphic involution. This identifies $Z^1(\Gamma, \text{Aut}(G)) \simeq Ant$, equivariantly under $\text{Aut}(G)$, while $\Hom(\mathbb Z/2, \text{Aut}(G)) = Hol$. The rest of the assertions now follow by descending the distinguished $G$-orbit (which is also an $\text{Aut}(G)$-orbit) on $Ant \times Hol$. Note that $\tau \in Ant$ corresponds to the trivial involution in $Hol$, hence these bijections are indeed pointed.
\end{proof}

\begin{remark}
\label{remark-inner-classes}
Two $1$-cocycles of $\Gamma$ into $\text{Aut}(G)$ are conjugate by $\text{Inn}(G)$ if and only if they correspond to the same inner class, see \ref{definition-pure-inner-form}. Thus, the quotient $Z^1(\Gamma, \text{Aut}(G))/\text{Inn}(G)$ is the union of all isomorphism classes of inner forms of all outer forms of $G$. 
\end{remark}




\begin{multicols}{2}[\section{Other chapters}]
\noindent
\begin{enumerate}
\item \hyperref[introduction-section-phantom]{Introduction}
\item \hyperref[representationtheory-section-phantom]{Basic Representation Theory}
\item \hyperref[representations-compact-section-phantom]{Representations of compact groups}
\item \hyperref[liegroups-general-section-phantom]{Lie groups and Lie algebras: general properties}
\item \hyperref[liestructure-section-phantom]{Structure of finite-dimensional Lie algebras}
\item \hyperref[vermamodules-section-phantom]{Verma modules}
\item \hyperref[algebraicgroups-section-phantom]{Linear algebraic groups}
\item \hyperref[reductiveforms-section-phantom]{Forms and covers of reductive groups, and the $L$-group}
\item \hyperref[galoiscohomology-section-phantom]{Galois cohomology of linear algebraic groups}
\item \hyperref[representations-local-section-phantom]{Representations of reductive groups over local fields}
%\item \hyperref[gKmodules-section-phantom]{$(\mathfrak g, K)$-modules}
%\item \hyperref[asymptotics-section-phantom]{Asymptotics and the Langlands classification}
\item \hyperref[plancherel-section-phantom]{Plancherel formula: reduction to discrete spectra}
\item \hyperref[discreteseries-section-phantom]{Construction of discrete series}
\item \hyperref[automorphicspace-section-phantom]{The automorphic space}
%\item \hyperref[harmonicanalysis-section-phantom]{Harmonic analysis over local fields}
\item \hyperref[automorphicforms-section-phantom]{Automorphic forms}
%\item \hyperref[periods-section-phantom]{Periods, theta correspondence, related methods}
%\item \hyperref[traceformulalocal-section-phantom]{The trace formula: local aspects}
%\item \hyperref[traceformulaglobal-section-phantom]{The trace formula: global aspects}
%\item \hyperref[arithmetic-section-phantom]{Arithmetic, reciprocity, Shimura varieties}
%\item \hyperref[geometric-section-phantom]{Geometric aspects}
\item \hyperref[fdl-section-phantom]{GNU Free Documentation License}
\item \hyperref[index-section-phantom]{Auto Generated Index}
\end{enumerate}
\end{multicols}





\bibliography{my}
\bibliographystyle{amsalpha}

\end{document}


