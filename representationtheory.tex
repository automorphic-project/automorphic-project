\IfFileExists{stacks-project.cls}{%
\documentclass{stacks-project}
}{%
\documentclass{amsart}
}

% The following AMS packages are automatically loaded with
% the amsart documentclass:
%\usepackage{amsmath}
%\usepackage{amssymb}
%\usepackage{amsthm}

\usepackage{amssymb}

% For dealing with references we use the comment environment
\usepackage{verbatim}
\newenvironment{reference}{\comment}{\endcomment}
%\newenvironment{reference}{}{}
\newenvironment{slogan}{\comment}{\endcomment}
\newenvironment{history}{\comment}{\endcomment}

% For commutative diagrams you can use
% \usepackage{amscd}
\usepackage[all]{xy}

% We use 2cell for 2-commutative diagrams.
\xyoption{2cell}
\UseAllTwocells

% To put source file link in headers.
% Change "template.tex" to "this_filename.tex"
% \usepackage{fancyhdr}
% \pagestyle{fancy}
% \lhead{}
% \chead{}
% \rhead{Source file: \url{template.tex}}
% \lfoot{}
% \cfoot{\thepage}
% \rfoot{}
% \renewcommand{\headrulewidth}{0pt}
% \renewcommand{\footrulewidth}{0pt}
% \renewcommand{\headheight}{12pt}

\usepackage{multicol}

% For cross-file-references
\usepackage{xr-hyper}

% Package for hypertext links:
\usepackage{hyperref}

% For any local file, say "hello.tex" you want to link to please
% use \externaldocument[hello-]{hello}
\externaldocument[introduction-]{introduction}
\externaldocument[representationtheory-]{representationtheory}
\externaldocument[representations-compact-]{representations-compact}
\externaldocument[liegroups-general-]{liegroups-general}
\externaldocument[liestructure-]{liestructure} 
\externaldocument[reductiveforms-]{reductiveforms}
\externaldocument[vermamodules-]{vermamodules}
\externaldocument[gKmodules-]{gKmodules}
\externaldocument[asymptotics-]{asymptotics}
\externaldocument[plancherel-]{plancherel}
\externaldocument[discreteseries-]{discreteseries}
%\externaldocument[algebraicgroups-]{algebraicgroups} 
%\externaldocument[harmonicanalysis-]{harmonicanalysis} 
%\externaldocument[automorphicforms-]{automorphicforms}
%\externaldocument[periods-]{periods}
%\externaldocument[traceformulalocal-]{traceformulalocal}
%\externaldocument[traceformulaglobal-]{traceformulaglobal}
%\externaldocument[arithmetic-]{arithmetic}
%\externaldocument[geometric-]{geometric}
\externaldocument[fdl-]{fdl}
\externaldocument[index-]{index}

% Theorem environments.
%
\theoremstyle{plain}
\newtheorem{theorem}[subsection]{Theorem}
\newtheorem{proposition}[subsection]{Proposition}
\newtheorem{lemma}[subsection]{Lemma}

\theoremstyle{definition}
\newtheorem{definition}[subsection]{Definition}
\newtheorem{example}[subsection]{Example}
\newtheorem{exercise}[subsection]{Exercise}
\newtheorem{situation}[subsection]{Situation}

\theoremstyle{remark}
\newtheorem{remark}[subsection]{Remark}
\newtheorem{remarks}[subsection]{Remarks}

\numberwithin{equation}{subsection}

% Macros
%
\def\lim{\mathop{\rm lim}\nolimits}
\def\colim{\mathop{\rm colim}\nolimits}
\def\Spec{\mathop{\rm Spec}}
\def\Hom{\mathop{\rm Hom}\nolimits}
\def\SheafHom{\mathop{\mathcal{H}\!{\it om}}\nolimits}
\def\SheafExt{\mathop{\mathcal{E}\!{\it xt}}\nolimits}
\def\Sch{\textit{Sch}}
\def\Mor{\mathop{\rm Mor}\nolimits}
\def\Ob{\mathop{\rm Ob}\nolimits}
\def\Sh{\mathop{\textit{Sh}}\nolimits}
\def\NL{\mathop{N\!L}\nolimits}
\def\proetale{{pro\text{-}\acute{e}tale}}
\def\etale{{\acute{e}tale}}
\def\QCoh{\textit{QCoh}}
\def\Ker{\text{Ker}}
\def\Im{\text{Im}}
\def\Coker{\text{Coker}}
\def\Coim{\text{Coim}}

\def\eqref #1{(\ref{#1})}


% OK, start here.
%
\begin{document}

\title{Representation theory: general notions}


\maketitle

\phantomsection
\label{section-phantom}


\tableofcontents


\section{Conventions}
\label{section-conventions}

The definition of representations makes sense over an arbitrary field, but very soon we start working with measures, specializing to the complex numbers as the coefficient field.

\section{Representations}
\label{section-representations}

Let $G$ be a topological group. Topological vector spaces are taken over a 
topological field $k$ (which we fix). We denote by $\text{End}(V)$, 
$\text{Aut}(V)$ the sets of {\it continuous} endomorphisms, resp.\ automorphisms, of a topological vector space $V$.


\begin{definition}
\label{definition-representation}
A {\it representation} of $G$ is a pair $(\pi,V)$, where $V$ is a topological vector space $V$ over $k$, and $\pi$ is a 
homomorphism
$$\pi: G\to \text{Aut}(V),$$
with the property that the induced ``action'' map:
$$G\times V\to V,$$
$$(g,v)\mapsto \pi(g) v$$
is continuous.

Representations of $G$ on topological $k$-vector spaces form a category, with a morphism 
$$ (\pi_1, V_1)\to (\pi_2,V_2)$$
being a continuous linear map $V_1\to V_2$ which commutes with the action of $G$. 

A {\it subrepresentation} of $V$ is a {\it closed} subspace of $V$ which is stable under the action of $G$.

A representation is called {\it irreducible} if it does not contain any non-zero, proper subrepresentations. 

A representation is called (totally) {\it decomposable} if it is equal to the direct sum of irreducible subrepresentations. \footnote{For infinite-dimensional representations, other notions of decomposition, such as by {\it direct integrals}, are often more useful. They will be discussed later.} 
\end{definition}


\begin{remark}
\label{remark-continuity-operatortopology}
 We do not require the map $G\to \text{Aut}(V)$ to be continuous in any of the usual operator topologies (for example, the norm on bounded linear operators, when $V$ is a Banach space), because this would preclude some of the most natural representations. For example, the rotation representation of $G=$ the circle on $V=L^2(\mathbb R)$ is not a continuous map $G\to \text{Aut}(V)$ with respect to the Hilbert norm on bounded operators. This may seem troubling at first, but it will appear more natural when we talk about the action of group algebras, see Remark \ref{remark-continuity-algebras}.
\end{remark}




\section{Action of measures on the group}
\label{section-measures}

From now on we assume that the field $k$ of coefficients of the representation is the field $\mathbb C$ of complex numbers, and that the topological group $G$ is locally compact. 

Let $M(G)$ be the Banach space of finite, complex-valued measures on $G$. (``Measures'' will always mean Radon measures.) It is a Banach algebra under convolution. Convolution is, by definition, the push-forward of measures under the multiplication map $G\times G \to G$. We denote by $M_c(G)$ the subalgebra of compactly supported finite measures. If $dg$ is a left Haar measure on $G$, we will call a measure $\mu= f dg$ continuous, $L^1$, etc, if $f$ is a function with the same property.

If $(\pi, V)$ is a topological representation of $G$, we would like to extend the action of $G$ to an action of the algebra $M(G)$ of measures, or a subalgebra $A$ thereof, in such a way that the action of $g\in G$ will correspond to the action of the delta measure at $g$. We assume throughout that the continuous dual $V^*$ of $V$ separates points. Then, such an extension will be characterized by the property
\begin{equation}
\label{equation-action-measures}
\left<\pi(\mu)(v), v^*\right> = \int_G \left<\pi(g)v, v^*\right> \mu(g),
\end{equation}
for every $\mu\in A$, $v\in V$, $v^*\in V^*$. 

\begin{proposition}
\label{proposition-integral-lcs}
 Assume that $V$ is a locally convex topological vector space, and such that the closure of the convex hull of any compact set is compact. Then, for every $\mu\in M_c(G)$, $v\in V$ the vector $\pi(\mu)(v)$ characterized by \eqref{equation-action-measures} is defined, and the resulting map $M_c(G)\times V \to V$ is continuous, when $M_c(G)$ is given the inductive limit topology over compact subsets of $G$. 
\end{proposition}

Notice that by \cite[Theorem 3.20]{Rudin}, Fr\'echet spaces satisfy the conditions of the proposition.

\begin{proof}
 The existence of an integral $\pi(\mu)(v):= \int_{G} \pi(g)(v) \mu(g)$ with the property \eqref{equation-action-measures}, for $\mu$ compactly supported, follows from \cite[Theorem 3.27]{Rudin}. 
 
 To show continuity, it is enough to show that, for every compact $K\subset G$, the preimage of a convex open neighborhood $U$ of $0$ contains an open neighborhood of $0$ in $M_c(K)\times V$. Recall that, by definition, the map $G\times V\to V$ is continuous. Thus, given a measure $K$ and an open $U\subset V$,  there is $U'\subset V$ open such that $\pi(g)(U')\subset U$ for every $g\in K$. If $\mu$ is a positive probability measure, by {\it loc.cit.\ }the integral $\int_{G} \pi(g)(v) \mu(g)$ belongs to the closure of the convex hull of the compact set $\pi(\text{supp} \mu)(v)$. Thus, for $v\in U'$, we will have $\pi(g)(v) \in \overline{U} \subset (1+\epsilon) U$ for every $\epsilon >0$; equivalently, $\pi(g)( (1+\epsilon)^{-1} U') \subset U$. Without loss of generality, the sets  $U'$ and $U$ are invariant under multiplication by the unit disk in $\mathbb C$, and then the preimage of $U$ contains the product with $(1+\epsilon)^{-1} U'$ of the open set of measures in $K$ which can be written as convex linear combinations of measures of the form $z\cdot \mu$, $|z|<1$.
\end{proof}




\begin{example}
\label{example-groupalgebra}
For a discrete group $G$, the convolution algebra $M_c(G)$ makes sense with arbitrary ring $k$ of coefficients, and coincides the {\it group algebra} $k[G]$. 

In this case, there is an (obvious) equivalence of categories between representations of $G$ on $k$-vector spaces (without topology, i.e., with the discrete topology) and $k[G]$-modules.
\end{example}


\begin{example}
\label{example-groupalgebra-Z}
In particular, for the group $G=\mathbb Z$, its group algebra is $k[T,T^{-1}]$, and its finitely generated representations (without topology) are classified by the structure theorem for finitely generated modules over principal ideal domains.
\end{example}





 
 
\section{Banach representations of compactly generated groups}
\label{section-Banach-representations}

\begin{definition}
\label{definition-radial-function}
 A {\it radial function} $r: G\to \mathbb R_+$, in the language of 
 \cite{Bernstein-Plancherel}
is a locally bounded positive function on $G$ such that $r(g_1\cdot g_2) \le r(g_1) + r(g_2)$. 

Two radial functions $r, r'$ are said to be {\it equivalent} if $(r+1)$ is comparable to $(r'+1)$, that is, there is a constant $C>0$ such that $C^{-1} (r+1)\le (r'+1) \le C (r+1)$.
\end{definition}

Suppose that $G$ is compactly generated and locally compact. Then there is a canonical equivalence class of radial functions on $G$, a representative of which is described by taking a compact generating subset $K$ with non-empty interior, and setting $r(g) = \min\{k| g\in K^k\}$. We will be working with such radial functions unless otherwise stated, and calling them ``natural''. 


\begin{lemma}
\label{lemma-bounded-by-radial}
Let $G$ be a compactly generated group, and $(\pi,V)$ a Banach representation of $G$.
There is a constant $C\ge 1$, depending on $V$ and the choice of radial function $r$, such that $\Vert \pi(g)\Vert \le C^{r(g)}$.
\end{lemma}

\begin{proof}
 From the definitions, if $K$ is a compact generating subset, defining the scale function $r$, then there is a constant $C$ such that $\Vert\pi(g)\Vert \le C$ for every $g\in K$, and therefore $\Vert\pi(g)\Vert \le C^{r(g)}$ for every $g\in G$.
\end{proof}

\begin{proposition}
\label{proposition-integral-Banach}
Let $G$ be a compactly generated, locally compact group, and $(\pi, V)$ a Banach representation of $G$. Endow $\text{End}(V)$ with the operator norm. The map: $M_c(G)\to \text{End}(V)$ is continuous, and bounded by the norm $\mu\mapsto \Vert \mu \cdot C^r\Vert$ on $M_c(G)$, for some natural radial function $r$ and constant $C \ge 1$.
\end{proposition}

\begin{proof}
 By \cite[Theorem 3.29]{Rudin}, 
 $$
 \left\Vert \int_G \pi(g)(v) \mu(g)\right\Vert \le \int_G \Vert \pi(g)\Vert |\mu|(g),
 $$
 and by Lemma \ref{lemma-bounded-by-radial} this is $\le \int_G C^{r(g)} |\mu|(g)$. 
\end{proof}


\begin{remark}
\label{remark-continuity-algebras}
Returning to Remark \ref{remark-continuity-operatortopology},  this proposition shows why it is not natural to require from the map $G\to \text{Aut}(V)\hookrightarrow \text{End}(V)$ to be continuous in the norm topology for $\text{End}(V)$: We can identify the action of elements $g\in G$ with the action of the corresponding delta measures, but in the space of measures we do not have $\delta_{g_n}\to \delta_g$ when $g_n\to g$.
\end{remark}


\begin{definition}
\label{definition-rapidly-decaying}
 Let $G$ be a compactly generated, locally compact group. The algebra $M_{rd}(G)$ of {\it rapidly decaying} measures on $G$ is the Fr\'echet subalgebra of $M(G)$ defined by the norms $\Vert \mu \cdot C^r\Vert$, for a natural radial function $r$ and all $C \ge 1$. 
\end{definition}

\begin{proposition}
\label{proposition-rapiddecay-Banach}
Every Banach representation extends to a continuous homomorphism $M_{rd}(G)\to \text{End}(V)$. 
\end{proposition}

\begin{proof}
 Follows immediately from Proposition \ref{proposition-integral-Banach}.
\end{proof}

\section{Unitary representations}
\label{section-unitary-representations}

We continue assuming that the coefficient field is $\mathbb C$, and that the topological group $G$ is locally compact.


\begin{definition}
\label{definition-unitary-representation}
A {\it unitary} representation of $G$ is a representation of $G$ on a Hilbert space\footnote{We will assume throughout that Hilbert spaces are separable.} $V$ (over $\mathbb C$) which preserves the norm (i.e.\ $\pi$ has image in the subgroup of unitary transformations, $U(V)\subset \text{Aut}(V)$). 

A representation $(\pi,V)$ of $G$ on a topological vector space $V$ is {\it unitarizable} if $V$ admits a (continuous, positive definite) inner product such that the corresponding Hilbert space completion is unitary.
\end{definition}

If $V$ is a Hilbert space, the algebra $B(V)$ of bounded operators on $V$ is a {\it $C^*$-algebra}: a Banach algebra, with an involution $T\mapsto T^*$ (the adjoint of an operator), and the property that $\Vert T^*T\Vert = \Vert T\Vert^2$.

$C^*$-algebras play an important role in the analysis of unitary representations of a group. 
The map $M_c(G)\to B(V)$ defines, by pullback, a seminorm on $M_c(G)$. Assume that $G$ is a locally compact group, with right Haar measure $dg$. The restriction of all those seminorms to $C_c(G)dg\subset M_c(G)$ defines a norm:
$$ \Vert fdg \Vert_{C^*} = \sup_{(\pi,V)} \Vert \pi(fdg)\Vert,$$
where $(\pi,V)$ ranges over all unitary representations of $G$. 

Considering just the right regular representation $R$ of $G$ on $L^2(G,dg)$, we obtain another norm
$$ \Vert fdg \Vert_{C_r^*} = \Vert R(fdg)\Vert.$$

\begin{definition}
\label{definition-Cstar}
The completion of $C_c(G) dg$ with respect to the norm $\Vert\bullet\Vert_{C^*}$ is the {\it $C^*$-algebra of $G$}. Its completion with respect to $\Vert\bullet \Vert_{C_r^*}$ is the {\it reduced $C^*$-algebra of $G$}.
\end{definition}

\section{$F$-representations}
\label{section-Frepresentations}

We continue assuming that the coefficient field is $\mathbb C$. We also assume that the topological group $G$ is locally compact, and compactly generated. 

\begin{definition}
\label{definition-Frepresentation}
An {\it $F$-representation} of $G$, in the language of 
\cite{Bernstein-Kroetz}, 
is a countable (inverse) limit of Banach representations, that is, a representation on a Fr\'echet space $V$, such that $V$ admits an equivariant topological isomorphism 
$$ 
V = \lim_{\leftarrow} V_n,
$$
where the $V_n$'s are Banach representations of $G$.
\end{definition}
 
Note that this is {\it stronger} than just a Fr\'echet representation of $G$: In a Fr\'echet representation, for every (continuous) seminorm $p$, and for every compact $K\subset G$, there is a seminorm $q$ such that 
$$ p(\pi(g) (v)) \le q(v)$$
for $g\in K$, $v\in V$. For an $F$-representation, there is a complete system of seminorms $p_n$ such that we can take $q_n=c_{K,n}\cdot p$, where $c_{K,n}$ is a scalar that depends on $K$ and $n$.

Note that by Proposition \ref{proposition-rapiddecay-Banach}, the action of $M_c(G)$ on $V$ extends to the measures $M_{rd}(G)$ of rapid decay. 

Fix a (natural) radial function $r$, and let $\Vert g\Vert:= e^{r(g)}$. The following definition is due to \cite{Casselman-canonicalextensions}:

\begin{definition}
\label{definition-moderate-growth}
A Fr\'echet representation $(\pi,V)$ of $G$ is said to be of {\it moderate
growth} if for any (continuous) seminorm $p$ on $V$ there exists a seminorm $q$
and an integer $N > 0$ such that
$$
p(\pi(g)v) \le \Vert g\Vert^N q(v)
$$
for all $g \in G$.
\end{definition}

\begin{lemma}
\label{lemma-F-moderate-growth} 
 Let $(\pi, V)$ be a Fr\'echet representation of the Lie group
$G$. Then the following statements are equivalent:
\begin{enumerate}
 \item  $(\pi,V)$ is of moderate growth;
 \item $(\pi, V)$ is an F-representation.
\end{enumerate}
\end{lemma}

\begin{proof}
See \cite[Lemma 2.10]{Bernstein-Kroetz}.
\end{proof}

\begin{remark}
\label{remark-generalscalefunctions}
 Bernstein and Kr\"otz introduce a more general notion of $F$-representations, which allows for more general scale functions on the group than the one defined in \ref{section-Banach-representations}.
\end{remark}






%**************************************************************************************


%*************************************************************************





\begin{multicols}{2}[\section{Other chapters}]
\noindent
Preliminaries
\begin{enumerate}
\item \hyperref[introduction-section-phantom]{Introduction}
\item \hyperref[conventions-section-phantom]{Conventions}
\item \hyperref[fdl-section-phantom]{GNU Free Documentation License}
\item \hyperref[index-section-phantom]{Auto Generated Index}
\end{enumerate}
\end{multicols}



\bibliography{my}
\bibliographystyle{amsalpha}

\end{document}
