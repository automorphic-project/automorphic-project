\IfFileExists{stacks-project.cls}{%
\documentclass{stacks-project}
}{%
\documentclass{amsart}
}

% The following AMS packages are automatically loaded with
% the amsart documentclass:
%\usepackage{amsmath}
%\usepackage{amssymb}
%\usepackage{amsthm}

\usepackage{amssymb}

% For dealing with references we use the comment environment
\usepackage{verbatim}
\newenvironment{reference}{\comment}{\endcomment}
%\newenvironment{reference}{}{}
\newenvironment{slogan}{\comment}{\endcomment}
\newenvironment{history}{\comment}{\endcomment}

% For commutative diagrams you can use
% \usepackage{amscd}
\usepackage[all]{xy}

% We use 2cell for 2-commutative diagrams.
\xyoption{2cell}
\UseAllTwocells

% To put source file link in headers.
% Change "template.tex" to "this_filename.tex"
% \usepackage{fancyhdr}
% \pagestyle{fancy}
% \lhead{}
% \chead{}
% \rhead{Source file: \url{template.tex}}
% \lfoot{}
% \cfoot{\thepage}
% \rfoot{}
% \renewcommand{\headrulewidth}{0pt}
% \renewcommand{\footrulewidth}{0pt}
% \renewcommand{\headheight}{12pt}

\usepackage{multicol}

% For cross-file-references
\usepackage{xr-hyper}

% Package for hypertext links:
\usepackage{hyperref}

% For any local file, say "hello.tex" you want to link to please
% use \externaldocument[hello-]{hello}
\externaldocument[introduction-]{introduction}
\externaldocument[representationtheory-]{representationtheory}
\externaldocument[representations-compact-]{representations-compact}
\externaldocument[liegroups-general-]{liegroups-general}
\externaldocument[liestructure-]{liestructure} 
\externaldocument[vermamodules-]{vermamodules}
\externaldocument[algebraicgroups-]{algebraicgroups}
\externaldocument[reductiveforms-]{reductiveforms}
\externaldocument[galoiscohomology-]{galoiscohomology}
\externaldocument[representations-local-]{representations-local}
%\externaldocument[gKmodules-]{gKmodules}
%\externaldocument[asymptotics-]{asymptotics}
\externaldocument[plancherel-]{plancherel}
\externaldocument[discreteseries-]{discreteseries}
\externaldocument[automorphicspace-]{automorphicspace}
%\externaldocument[harmonicanalysis-]{harmonicanalysis} 
\externaldocument[automorphicforms-]{automorphicforms}
%\externaldocument[periods-]{periods}
%\externaldocument[traceformulalocal-]{traceformulalocal}
%\externaldocument[traceformulaglobal-]{traceformulaglobal}
%\externaldocument[arithmetic-]{arithmetic}
%\externaldocument[geometric-]{geometric}
\externaldocument[fdl-]{fdl}
\externaldocument[index-]{index}

% Theorem environments.
%
\theoremstyle{plain}
\newtheorem{theorem}[subsection]{Theorem}
\newtheorem{proposition}[subsection]{Proposition}
\newtheorem{lemma}[subsection]{Lemma}

\theoremstyle{definition}
\newtheorem{definition}[subsection]{Definition}
\newtheorem{example}[subsection]{Example}
\newtheorem{exercise}[subsection]{Exercise}
\newtheorem{situation}[subsection]{Situation}

\theoremstyle{remark}
\newtheorem{remark}[subsection]{Remark}
\newtheorem{remarks}[subsection]{Remarks}

\numberwithin{equation}{subsection}

% Macros
%
\def\lim{\mathop{\rm lim}\nolimits}
\def\colim{\mathop{\rm colim}\nolimits}
\def\Spec{\mathop{\rm Spec}}
\def\Hom{\mathop{\rm Hom}\nolimits}
\def\SheafHom{\mathop{\mathcal{H}\!{\it om}}\nolimits}
\def\SheafExt{\mathop{\mathcal{E}\!{\it xt}}\nolimits}
\def\Sch{\textit{Sch}}
\def\Mor{\mathop{\rm Mor}\nolimits}
\def\Ob{\mathop{\rm Ob}\nolimits}
\def\Sh{\mathop{\textit{Sh}}\nolimits}
\def\NL{\mathop{N\!L}\nolimits}
\def\proetale{{pro\text{-}\acute{e}tale}}
\def\etale{{\acute{e}tale}}
\def\QCoh{\textit{QCoh}}
\def\Ker{\text{Ker}}
\def\Im{\text{Im}}
\def\Coker{\text{Coker}}
\def\Coim{\text{Coim}}

\def\eqref #1{(\ref{#1})}
\newcommand{\sslash}{\mathbin{/\mkern-6mu/}}



% OK, start here.
%
\begin{document}

\title{Galois cohomology of linear algebraic groups}


\maketitle

\phantomsection
\label{section-phantom}

\tableofcontents

[This chapter needs a lot of work. For now, we only summarize the results needed in other sections.]

Let $G$ be a linear algebraic group over a field $k$ and $k^{s}$ be a separable closure of $k$. There is a natural action of $Gal (k^{s}/k)$ on the $k^{s}$-points of $G$, and we can define the Galois cohomology group $H^{1}(k, G)$. In this chapter, we discuss this cohomology group for $k$ is finite, local, or number field. A good reference for this chapter is Chapter 6 of \cite{Platonov-Rapinchuk}.

\section{Galois cohomology over a finite field}
\label{section-finite-field}


In this section, $k$ is a finite field. The absolute Galois group $Gal(k^{s}/k)\cong \hat{\mathbb{Z}}$ is a procyclic group. Let $\varphi$ be the arithmetic Frobenius in $Gal(k^{s}/k)$. Then $\varphi$ is a topological generator of $Gal(k^{s}/k)$ and it induces a $k$-scheme endomorphism $\varphi_{G}:=id_{G}\times \varphi$ of $G_{k^{s}}$.
\begin{theorem}[Lang's theorem]
 \label{theorem-Lang}
If $G$ is a connected algebraic group over $k$, then $H^1(k, G)=1$. 
\end{theorem}

\begin{proof}
We define a $k$-scheme morphism $f: G\rightarrow G$ by  
\[
f(g)=g^{-1}\varphi (g).
\]
To prove Lang's theorem, it suffices to prove that $f$ is surjective. 

Considering the action of $G$ on itself by $g.a=g^{-1}a\varphi(g)$. If we fix $a$, this define a $k$-scheme endomorphism of $G$ denoted by $f_{a}$. In particular, $f_{e}=f$. We claim that the map $f_{a}$ is separable and its image is open and closed for any $a$. Then $f$ is surjective since $G$ is connected. This would complete the proof of Lang's theorem.

Let me prove the claim. We have 
\[
d_{e}f_{a}(X)=-Xa +d_{e}\varphi_{G}(X)=-Xa
\]
for $X\in T_{e}(G)$. So the differential map $d_{e}f_{a}: T_{e}(G)\rightarrow T_{a}(G)$ is an isomorphism of the tangent spaces. As a consequence, $f_{a}$ is dominant and separable. In particular, the orbit $f_{a}(G)$ contains a nonempty open subset of $G$, hence is open by homogeneity. Since this holds for any $a$, $f_{a}(G)$ are also closed. We are done.
\end{proof}


Lang's theorem has several important corollaries.
\begin{proposition}
 \label{proposition-quasisplit}
If $G$ is a connected reductive group over $k$, then $G$ is quasisplit.
\end{proposition}
\begin{proof}
 Let $B$ be a Borel subgroup of $G_{k^{s}}$, and let $B^{\varphi}$ be the Borel subgroup obtained by applying $\varphi$. Then $aB^{\varphi}a^{-1}=B$ for some $a$ in $G(k^{s})$. By the proof of Theorem \ref{theorem-Lang}, $a=g^{-1}\varphi (g)$ for some $g$ in $G(k^{s})$. Now, we consider the Borel subgroup $H=gBg^{-1}$. We can check that $H$ is defined over $k$ by the following computation.
\[
H^{\varphi}=\varphi(g)B^{\varphi} \varphi(g)^{-1}=gaB^{\varphi}a^{-1}g^{-1}=H.
\]
This completes the proof of this proposition.

\end{proof}

\begin{proposition}[Lang's isogeny theorem]
 \label{proposition-Lang-isogeny}
If $G$ and $H$ are connected $k$-groups and $f: G\rightarrow H$ is a $k$-isogeny, then $|G(k)|=|H(k)|$.
\end{proposition}

\begin{proof}
Let $F$ be the kernel of $f $. We have a short exact sequence
\[
1\rightarrow F(k^{s})\rightarrow G(k^{s})\rightarrow H(k^{s})\rightarrow 1
\]
of groups. This exact sequence is compatible with the natural Galois action. Then we obtain an exact sequence 
\[
\{\ast\}\rightarrow F(k)\rightarrow G(k)\rightarrow H(k)\rightarrow H^{1}(k, F)\rightarrow H^{1}(k, G)
\]
of pointed set. By Theorem \ref{theorem-Lang}, we have 
\[
\frac{|H(k)|}{|G(k)|}=\frac{|H^{1}(k, F)|}{|F(k)|}.  
\]
So it suffices to show that $|H^{1}(k, F)|=|F(k)|$. Note that 
\[
|H^{1}(k, F)|=\varinjlim H^1(Gal (k_{n}/k), F(k_{n}))
\]
where $k_n$ is the degree $n$ extension of $k$. So it is enough to prove that, for each $n$, $|H^1(Gal (k_{n}/k), F(k_{n}))|=|F(k)|$. This is a property of Herbrand quotient (see \cite[Proposition 11]{Atiyah-Wall}). This completes the proof.
\end{proof}

\begin{remark}
 \label{remark-classify-group-finite}
Lang's theorem can also be used to classify connected reductive groups over a finite field. Let $G$ be a split connected reductive group over $k$.  We fix a based root system $\Psi^{+}$ of $G$. By Theorem \ref{theorem-Lang}, we have $H^{1}(k, Aut (G_{k^{s}}))\cong H^{1}(k, Aut (\Psi^{+}))$. So the $k$-forms of $G$ are classified by the elements of $H^{1}(k, Aut(\Psi^{+}))$. We have the following conclusions.
\begin{enumerate}
\item[(1)] Any $k$-group of type $B_n$, $C_n$, $E_7$, $E_8$, $F_4$ or $G_2$ is split.
\item[(2)] There are exactly two nonisomorphic $k$-groups of type $A_n$ (where $n>1$) and $D_n$ (where $n>4$). The nonsplit one is split over a quadratic extension of $k$. 
\item[(3)] There are exactly three nonisomorphic $k$-groups. The two non-split ones become split over a quadratic and a cubic extension of $k$ respectively.
\end{enumerate}
\end{remark}


\begin{theorem}
 \label{theorem-almost-quasisplit}
If $G$ is a connected algebraic group over a number field $K$, then $G_{K_{v}}$ is quasisplit for almost all finite places $v$ of $K$.
\end{theorem}

\begin{proof}
See \cite[Theorem 6.7]{Platonov-Rapinchuk}.
\end{proof}

Using Lang's theorem we can also deduce a result on Galois cohomology of groups over the ring of integers of a local field. Let $K$ be a local field with ring of integers $\mathcal{O}_{K}$. Let $G_{\mathcal{O}_{K}}$ be an algebraic group defined over $\mathcal{O}_{K}$ and let $L$ be a finite Galois extension of $K$. The Galois group $Gal(L/K)$ acts naturally on the $\mathcal{O}_{L}$-point of $G_{\mathcal{O}_{K}}$. We can define the Galois cohomology group $H^{1}(L/K, G_{\mathcal{O}_{K}})$.
\begin{theorem}
 \label{theorem-unramified-cohomology}
If a connected group $G_{\mathcal{O}_{K}}$ has a connected smooth reduction $\underline{G_{\mathcal{O}_{K}}}$ and the extension $L/K$ is unramified, then $H^{1}(L/K, G_{\mathcal{O}_{K}})=1$.
\end{theorem}

\begin{proof}
See \cite[Theorem 6.8]{Platonov-Rapinchuk}.
\end{proof}

\begin{remark}
\label{remark-trivial-almost-everywhere}
Let $L$ be a finite extension of a number field $K$. By the above theorem, if $G$ is a connected algebraic group over $K$, then for almost all finite places $v$ of $K$ and any $w$ a place of $L$ above $v$, we have $H^{1}(L_{w}/K_{v}, G_{\mathcal{O}_{K_{v}}})=1$. 
\end{remark}



\section{Tate--Nakayama duality for tori}
 \label{section-Tate-Nakayama}
 

\section{Cohomology of reductive groups over local fields}
\label{section-local-fields}

\begin{lemma}
 \label{lemma-cohomology-finite}
If $G$ is an algebraic group over a local field $F$, then $H^1(F, G)$ is finite.
\end{lemma}

\begin{proof}
 
\end{proof}


\begin{theorem}
 \label{theorem-H1-trivial}
If $G$ is a (connected) simply connected, semisimple group over a non-Archimedean field $F$, then $H^1(F,G)$ is trivial. For an arbitrary connected reductive group over a local field $F$, there is a canonical surjective map [Kottwitz], which in the non-Archimedean case is a bijection.
\end{theorem}

\begin{proof}
 
\end{proof}

\section{Cohomology of reductive groups over global fields; the Hasse principle}
\label{section-global-fields}

\begin{definition}
 \label{definition-Sha-Hasse}
Let $G$ be an algebraic group over a global field $k$. The kernel of the natural map $H^1(k,G)\to \prod_v H^1(k_v, G)$ is the {\it Tate--Shafarevich group} of $G$, denoted $\text{Sha}(G)$. We say that $G$ satisfies the {\it Hasse principle} if $\text{Sha}(G)=1$.
\end{definition}



\begin{theorem}
 \label{theorem-Sha-finite}
 If $G$ is an algebraic group over a number field, then $\text{Sha}(G)$ is finite.
\end{theorem}

\begin{proof}
 
\end{proof}


The following is the Hasse principle for algebraic groups, due to Kneser, Harder (who proved it for groups without $E_8$ factors)  and Chernousov (who completed the $E_8$ case) over number fields, and to Harder over function fields.

\begin{theorem}
 \label{theorem-Hasse-principle}
If $G$ is (connected and) simply connected or adjoint over a global field, then $\text{Sha}(G) =1$.
\end{theorem}


\begin{proof}
 See \cite[Theorems 6.6]{Platonov-Rapinchuk} for the number field case, and \cite[Theorems 6.22]{Platonov-Rapinchuk} for the reduction of the adjoint case to the simply connected case. The proof for number fields involves a difficult case-to-case analysis. For function fields, there is a general proof due to Harder, \cite{Harder-Galoiskohomologie}.
\end{proof}

\begin{proposition}
 \label{proposition-H1-surjects}
If $G$ is a connected algebraic group over a global field $k$, then $H^1(k,G) \to \prod_{v|\infty} H^1(k_v, G)$ is surjective.
\end{proposition}


\begin{proof}
 See \cite[Proposition 6.17]{Platonov-Rapinchuk}.
\end{proof}






%***************************************************************************




\begin{multicols}{2}[\section{Other chapters}]
\noindent
\begin{enumerate}
\item \hyperref[introduction-section-phantom]{Introduction}
\item \hyperref[representationtheory-section-phantom]{Basic Representation Theory}
\item \hyperref[representations-compact-section-phantom]{Representations of compact groups}
\item \hyperref[liegroups-general-section-phantom]{Lie groups and Lie algebras: general properties}
\item \hyperref[liestructure-section-phantom]{Structure of finite-dimensional Lie algebras}
\item \hyperref[vermamodules-section-phantom]{Verma modules}
\item \hyperref[algebraicgroups-section-phantom]{Linear algebraic groups}
\item \hyperref[reductiveforms-section-phantom]{Forms and covers of reductive groups, and the $L$-group}
\item \hyperref[galoiscohomology-section-phantom]{Galois cohomology of linear algebraic groups}
\item \hyperref[representations-local-section-phantom]{Representations of reductive groups over local fields}
%\item \hyperref[gKmodules-section-phantom]{$(\mathfrak g, K)$-modules}
%\item \hyperref[asymptotics-section-phantom]{Asymptotics and the Langlands classification}
\item \hyperref[plancherel-section-phantom]{Plancherel formula: reduction to discrete spectra}
\item \hyperref[discreteseries-section-phantom]{Construction of discrete series}
\item \hyperref[automorphicspace-section-phantom]{The automorphic space}
%\item \hyperref[harmonicanalysis-section-phantom]{Harmonic analysis over local fields}
\item \hyperref[automorphicforms-section-phantom]{Automorphic forms}
%\item \hyperref[periods-section-phantom]{Periods, theta correspondence, related methods}
%\item \hyperref[traceformulalocal-section-phantom]{The trace formula: local aspects}
%\item \hyperref[traceformulaglobal-section-phantom]{The trace formula: global aspects}
%\item \hyperref[arithmetic-section-phantom]{Arithmetic, reciprocity, Shimura varieties}
%\item \hyperref[geometric-section-phantom]{Geometric aspects}
\item \hyperref[fdl-section-phantom]{GNU Free Documentation License}
\item \hyperref[index-section-phantom]{Auto Generated Index}
\end{enumerate}
\end{multicols}





\bibliography{my}
\bibliographystyle{amsalpha}

\end{document}
