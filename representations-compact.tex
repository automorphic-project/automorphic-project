\IfFileExists{stacks-project.cls}{%
\documentclass{stacks-project}
}{%
\documentclass{amsart}
}

% The following AMS packages are automatically loaded with
% the amsart documentclass:
%\usepackage{amsmath}
%\usepackage{amssymb}
%\usepackage{amsthm}

\usepackage{amssymb}

% For dealing with references we use the comment environment
\usepackage{verbatim}
\newenvironment{reference}{\comment}{\endcomment}
%\newenvironment{reference}{}{}
\newenvironment{slogan}{\comment}{\endcomment}
\newenvironment{history}{\comment}{\endcomment}

% For commutative diagrams you can use
% \usepackage{amscd}
\usepackage[all]{xy}

% We use 2cell for 2-commutative diagrams.
\xyoption{2cell}
\UseAllTwocells

% To put source file link in headers.
% Change "template.tex" to "this_filename.tex"
% \usepackage{fancyhdr}
% \pagestyle{fancy}
% \lhead{}
% \chead{}
% \rhead{Source file: \url{template.tex}}
% \lfoot{}
% \cfoot{\thepage}
% \rfoot{}
% \renewcommand{\headrulewidth}{0pt}
% \renewcommand{\footrulewidth}{0pt}
% \renewcommand{\headheight}{12pt}

\usepackage{multicol}

% For cross-file-references
\usepackage{xr-hyper}

% Package for hypertext links:
\usepackage{hyperref}

% For any local file, say "hello.tex" you want to link to please
% use \externaldocument[hello-]{hello}
\externaldocument[introduction-]{introduction}
\externaldocument[representationtheory-]{representationtheory}
\externaldocument[representations-compact-]{representations-compact}
\externaldocument[liegroups-general-]{liegroups-general}
\externaldocument[liestructure-]{liestructure} 
\externaldocument[reductiveforms-]{reductiveforms}
\externaldocument[vermamodules-]{vermamodules}
\externaldocument[gKmodules-]{gKmodules}
\externaldocument[asymptotics-]{asymptotics}
\externaldocument[plancherel-]{plancherel}
\externaldocument[discreteseries-]{discreteseries}
%\externaldocument[algebraicgroups-]{algebraicgroups} 
%\externaldocument[harmonicanalysis-]{harmonicanalysis} 
%\externaldocument[automorphicforms-]{automorphicforms}
%\externaldocument[periods-]{periods}
%\externaldocument[traceformulalocal-]{traceformulalocal}
%\externaldocument[traceformulaglobal-]{traceformulaglobal}
%\externaldocument[arithmetic-]{arithmetic}
%\externaldocument[geometric-]{geometric}
\externaldocument[fdl-]{fdl}
\externaldocument[index-]{index}

% Theorem environments.
%
\theoremstyle{plain}
\newtheorem{theorem}[subsection]{Theorem}
\newtheorem{proposition}[subsection]{Proposition}
\newtheorem{lemma}[subsection]{Lemma}

\theoremstyle{definition}
\newtheorem{definition}[subsection]{Definition}
\newtheorem{example}[subsection]{Example}
\newtheorem{exercise}[subsection]{Exercise}
\newtheorem{situation}[subsection]{Situation}

\theoremstyle{remark}
\newtheorem{remark}[subsection]{Remark}
\newtheorem{remarks}[subsection]{Remarks}

\numberwithin{equation}{subsection}

% Macros
%
\def\lim{\mathop{\rm lim}\nolimits}
\def\colim{\mathop{\rm colim}\nolimits}
\def\Spec{\mathop{\rm Spec}}
\def\Hom{\mathop{\rm Hom}\nolimits}
\def\SheafHom{\mathop{\mathcal{H}\!{\it om}}\nolimits}
\def\SheafExt{\mathop{\mathcal{E}\!{\it xt}}\nolimits}
\def\Sch{\textit{Sch}}
\def\Mor{\mathop{\rm Mor}\nolimits}
\def\Ob{\mathop{\rm Ob}\nolimits}
\def\Sh{\mathop{\textit{Sh}}\nolimits}
\def\NL{\mathop{N\!L}\nolimits}
\def\proetale{{pro\text{-}\acute{e}tale}}
\def\etale{{\acute{e}tale}}
\def\QCoh{\textit{QCoh}}
\def\Ker{\text{Ker}}
\def\Im{\text{Im}}
\def\Coker{\text{Coker}}
\def\Coim{\text{Coim}}

\def\eqref #1{(\ref{#1})}


% OK, start here.
%
\begin{document}

\title{Representations of compact groups}


\maketitle

\phantomsection
\label{section-phantom}


\tableofcontents


The (complex) representation theory of compact groups, in a nutshell:
\begin{enumerate}
 \item All representations are unitarizable, by averaging an inner product against Haar measure.
 \item All unitary representations decompose into (Hilbert space, i.e., orthogonal and completed) direct sums of finite-dimensional irreducibles (the Peter--Weyl theorem), because operators of convolution by continuous measures are Hilbert--Schmidt (hence compact).
\end{enumerate}


In this chapter, $G$ is a compact group, and we fix throughout the probability Haar measure $dg$. Multiplication of functions by $dg$ turns them into measures which can act on the space of a representation $\pi$, and we feel free to write $\pi(f)$ for $\pi(f dg)$. 

\section{Unitarity}
\label{section-unitarity}

\begin{proposition}
\label{proposition-compact-unitarizable}
 Let $(\pi,V)$ be a representation of $G$ on a space admitting an inner product (positive definite hermitian form). Then, it is unitarizable.
\end{proposition}

\begin{proof}
 Take any positive definite hermitian form $\left< \, , \,\right>'$, and integrate it over the action of the group in order to make it invariant:
 $$ \left<  v, v\right>:= \int_G \left< \pi(g) v, \pi(g) v\right>' dg.$$
\end{proof}

\section{Hilbert--Schmidt property of continuous convolution operators}
\label{section-Hilbert-Schmidt}

Given a Hilbert space $V$, its linear dual $V^*$ is identified with its complex conjugate $\bar V$. We denote by $B(V)=\text{End}(V)$ the space of bounded linear operators on $V$, and by $\hat\otimes$ the Hilbert space tensor product of two Hilbert spaces, i.e., the completion of the algebraic tensor product with respect to the Hilbert norm characterized by $\Vert v \otimes w\Vert = \Vert v\Vert\cdot \Vert w \Vert$.

There is a natural embedding $\bar V\hat\otimes V \hookrightarrow B(V)$, with $\bar v_1 \otimes v_2$ mapping 
to the operator 
$ w\mapsto \left<  w, v_1 \right> \cdot v_2$.
The image is the space of {\it Hilbert--Schmidt operators}, and the Hilbert norm on $\bar V\hat\otimes V$ is the Hilbert--Schmidt norm. Hilbert--Schmidt operators are {\it compact}: they map bounded sets to precompact sets. (It is an easy exercise to prove this, using the explicit expression $\Vert T\Vert_{HS}^2 = \sum_i \Vert Te_i\Vert^2$.)


\begin{proposition}
\label{proposition-continuousconvolution-compact}
 Let $L$ (resp.\ $R$) denote the left (resp.\ right) regular representation of $G$ on $L^2(G)$. For every continuous measure $\mu$ on $G$ the operator $L(\mu)$ (resp.\ $R(\mu)$) is Hilbert-Schmidt and, hence, compact.
\end{proposition}

\begin{proof}
 Let $\mu=h dg$. Then the operator $L(\mu)$ has the integral expression:
$$ L(\mu)(f)(x) = \int_{H} K_h(x,y) f(y) dy,$$
where the kernel is given by:
$$ K_h(x,y) = h(xy^{-1}) dh.$$

The Hilbert Schmidt norm of an integral operator $T$ with kernel $K$ on a measure space $(X,dx)$ is given by:
$$ \Vert T\Vert_{HS}^2= \Vert K\Vert^2_{L^2(X\times X)}.$$
In particular, $L(\mu)$ is Hilbert-Schmidt. (It was important here that the group was compact for the $L^2$-norm of $K_h$ to be finite.) 
\end{proof}



\section{Recollection of spectral theorems}
\label{section-recollection-spectraltheorems}

We recall the following spectral theorems from functional analysis. 

Let $V$ be a Hilbert space. The \emph{adjoint} of a (bounded) operator $T$ on $V$ is the operator $T^*$ with $\left<  T^*v,w\right>=\left<  v,Tw\right>$. An operator $T$ is called \emph{normal} if $T^* T = T T^*$, and \emph{self-adjoint} if $T=T^*$. The \emph{spectrum} $\sigma(T)$ of an operator $T$ is the (closed) subset of all $\lambda\in \mathbb C$ such that $T-\lambda I$ is not invertible. 

The idea of the spectral theorem is that the space $V$ decomposes as an ``integral'' of ``eigenspaces'' of $T$. A familiar case of an integral of eigenspaces is when $V=L^2(\mathbb R)$, in which case the theory of Fourier transform says:
$$V= \int_{i\mathbb R} \left< e^{sx}\right> ds,$$
with the spaces $\left< e^{sx}\right>$ being eigenspaces for all translation operators. 

\begin{theorem}[Spectral theorem for normal operators]
\label{theorem-spectraltheorem-normal}
 If $T$ is a normal operator on a Hilbert space $V$ then there is a measure space $(X,\mathcal B,\mu)$, a measurable function $\lambda: X\to \mathbb C$ and a unitary isomorphism: $V \simeq L^2(X,\mu)$ which carries the operator $T$ to ``multiplication by $\lambda$''.
 
 Moreover, for every measurable $\omega\subset \mathbb C$, let $E(\omega)$ be the projection (restriction) $L^2(X,\mu) \to L^2(\lambda^{-1}(\omega),\mu)$. Then, $E(\sigma(T))=I$, $E(\omega)\ne 0$ for every relatively open nonempty subset of $\sigma(T)$, and $E(\omega)$ commutes with the commutator of $T$ in $B(V)$.
\end{theorem}

\begin{proof}
 See \cite[Theorem 12.23]{Rudin}. The measure space space $(X,\mu)$ in our formulation is not canonical, but can be obtained from the more canonical formulation of \emph{loc.cit}.\ by considering subspaces of $V$ of the form $\overline{\mathbb C[T]v}$, for $v\in V$. 
\end{proof}



An operator on a Hilbert space is \emph{compact} if it can be approximated in the operator norm by operators with finite-dimensional range. Equivalently, if it maps bounded sets to precompact sets (i.e.\ sets whose closure is compact). 

\begin{theorem}
\label{theorem-spectraltheorem-compact} 
 Let $T$ be a compact self-adjoint operator on a Hilbert space $V$, then there is a sequence of eigenvectors $v_n$ with (real) eigenvalues $0\ne\lambda_n\to 0$ (or a finite number of such eigenvalues, if the space is finite-dimensional) such that 
 $$V = \ker T \oplus \hat{\bigoplus}_n \left< v_n\right>,$$
a Hilbert space (i.e., orthogonal and completed) direct sum. 
\end{theorem}

\begin{proof}
 See \cite[Theorem 12.30]{Rudin}.
\end{proof}


In particular, all eigenspaces with nonzero eigenvalues are finite-dimensional.


\begin{theorem}[Schur's lemma]
\label{theorem-Schurslemma}
 If $V, V'$ are two irreducible unitary representations of a group $G$, and $T \in \Hom^G(V,V')$, then $T$ is a scalar multiple of an isometry.
\end{theorem}

\begin{proof}
 It is enough to show that the self-adjoint bounded operators $T^* T$ and $T T^*$ are scalars. Let $S$ be one of these operators, and apply the spectral theorem. We claim that the spectrum $\sigma(S)$ is a singleton, hence making $S$ a scalar operator. Otherwise, by Theorem \ref{theorem-spectraltheorem-normal}, for two non-empty, disjoint open subsets of $\sigma(S)$, the corresponding projections $E(\omega_1)$ and $E(\omega_2)$ are non-zero, and have orthogonal images. But these projections commute with the action of $G$, which commutes with $T$, hence $V$ cannot be irreducible.
\end{proof}



\section{Peter--Weyl theorems} \label{PeterWeyl}
\label{section-PeterWeyl}

Let $H$ be a compact group, and consider the space $V=L^2(H)$. It is a unitary representation for $G=H\times H$. 

\begin{lemma}
\label{lemma-matrixcoefficient}
\begin{enumerate}
  \item For any finite-dimensional irreducible representation $\pi$ of $H$, we have an embedding $M_\pi: \pi^*\otimes\pi\to C(H) \subset L^2(H)$ given by the matrix coefficient map
  $$ \tilde v\otimes v \mapsto (h\mapsto \left< \pi(h) v, \tilde v\right>).$$
  \item For non-isomorphic $\pi$'s, the images of these embeddings are orthogonal.
 \end{enumerate}
\end{lemma}

\begin{proof}
 The matrix coefficient map is an embedding (injection), because it is clearly non-zero, and $\pi^*\otimes\pi$ is an irreducible representation of $G=H\times H$.
 
 If $\pi, \sigma$ are irreducible and non-isomorphic, the orthogonal projection from the image of $M_\pi$ to the image of $M_\sigma$ is $G$-equivariant, and since $\pi^*\otimes \pi$ is not isomorphic to $\sigma^*\otimes \sigma$, it has to be zero.
\end{proof}


Notice that the image of the matrix coefficient map consists of \emph{(left and right) finite} vectors (since $\pi\otimes\pi^*$ is finite dimensional). Our goal is to prove:

\begin{theorem}[Peter--Weyl theorem]
\label{theorem-PeterWeyl}
The matrix coefficient maps give rise to a canonical isomorphism
$$L^2(H)\simeq \hat\bigoplus_\pi \pi^*\otimes \pi$$
(Hilbert space direct sum),
where $\pi$ runs over representatives for the isomorphism classes of irreducible, finite dimensional representations of $H$.
\end{theorem}

\begin{proof}
 We prove the theorems in steps. 

\begin{itemize}
 \item Let $\mu_n$ be a sequence of approximations of the identity in $H$, i.e.\ positive probability measures whose mass is eventually concentrated in any neighborhood of the identity. Then for any Banach representation $(\pi,V)$ and vector $v\in V$ we have $\pi(\mu_n)v\to v$.
 
 Indeed, $\mu_n\to \delta_1$ in the measure space $M(H)$, and we have seen (Proposition \ref{representationtheory-proposition-integral-Banach}) that the map $\pi: M(H)\to \text{End}(V)$ is continuous.

 
 \item For any subrepresentation $V$ of $L^2(H)$ under the right (or left) regular action, continuous functions are dense in $V$. 

Indeed, it is enough to choose a continuous approximation of the identity and convolve elements of $V$ with it; the convolution of any function with a continuous measure is continuous.

\item Right-finite (or left-finite) functions are dense in any closed, invariant subspace of $L^2(H)$. 

This is the most important step of the proof. Assume to the contrary that there is a non-zero closed subspace $V$ without any right finite functions. We can find a continuous, self-adjoint measure $\mu$ on $H$ such that $L(\mu)V\ne 0$. (Indeed, we can do this by approximating the identity by continuous, self-adjoint measures; here by self-adjoint we mean that the operator $L(\mu)$ is self-adjoint, which is equivalent to $h(g^{-1})=\overline{h(g)}$ if $\mu=hdg$ --- exercise!.) We know (Proposition \ref{proposition-continuousconvolution-compact}) that $L(\mu)$ is compact, hence by the Spectral Theorem \ref{theorem-spectraltheorem-compact} there is a non-zero (real) eigenvalue $\lambda$ of $L(\mu)$, and the $\lambda$-eigenspace is finite-dimensional. But the $\lambda$-eigenspace for $L(\mu)$ is stable under the right action of $H$, hence there are right-finite vectors, a contradiction.

\end{itemize}

Now let $\pi$ be a finite-dimensional irreducible representation of $H$; by the previous point, such representations exist. We have a tautological map $T:\text{Hom}(\pi, L^2(H)) \otimes \pi\to L^2(H)$, where $\text{Hom}$ refers here to the right regular representation. Moreover, the image of $T$ lies in the subspace $C(H)$ of continuous functions. If we endow the space $\text{Hom}(\pi, L^2(H))$ with the action induced from the left regular representation of $H$ on $L^2(H)$, the map $T$ is equivariant. Evaluation at the identity defines a morphism $\text{Hom}(\pi, L^2(H)) \to \pi^*$, whose kernel is (tautologically) trivial. We conclude that the $\pi$-isotypic component of $L^2(H)$ under the right representation (the image of $T$) is isomorphic to $\pi^*\otimes \pi$. Those subspaces, as the isomorphism class of $\pi$ varies, are mutually orthogonal by Lemma \ref{lemma-matrixcoefficient}, and they span a dense subspace, by the points above. This proves the theorem.
\end{proof}


\begin{theorem}
\label{theorem-PeterWeyl-general}
 Every Fr\'echet representation of $H$ contains a dense subspace of finite vectors. In particular, every irreducible Fr\'echet represenation is finite dimensional. Every Hilbert representation of $H$ is the Hilbert space direct sum of irreducibles.
\end{theorem}


\begin{proof}

Choose an approximation of the identity by $L^2$-measures $f_ndh$. By Theorem \ref{theorem-PeterWeyl}, we can approximate $f_n$ in $L^2(H)$, and hence in $L^1(H)$, by left-finite functions, thus there is a sequence of left-finite measures $\mu_n$ such that $\mu_n\to \delta_1$. Then, for any vector $v\in V$ we have: $\pi(\mu_n)v\to v$. This follows from Proposition \ref{representationtheory-proposition-integral-Banach}, together with the (easy) fact that, for compact groups, every Fr\'echet representation is an $F$-representation, in the language of \ref{representationtheory-section-Frepresentations}.

But, for a vector $v\in V$, $\pi(\mu_n)v$ is finite since $\mu_n$ is left-finite. This proves the first claim, and the others follow easily.

\end{proof}




\begin{proposition}
\label{proposition-denseinclusions} 
Assume that $H$ is a compact Lie group (or just a compact group, ignoring mentions of smooth vectors below). We have a sequence of dense inclusions of Fr\'echet spaces:
\begin{equation}
 L^2(H)_{\text{fin}} \subset C^\infty(H) \subset C(H)\subset L^2(H),
\end{equation}
where $~_{\text{fin}}$ denotes left and right finite functions.
\end{proposition}

\begin{proof}
If $H$ is a Lie group, we can show as in the proof of Theorem \ref{theorem-PeterWeyl}, by choosing a smooth approximation of the identity, that any subrepresentation of $L^2(H)$ contains a dense subspace of smooth vectors, and that $L^2(H)_{\text{fin}}$ belongs to the space of smooth functions. 


We then apply Theorem \ref{theorem-PeterWeyl-general} to any of these Fr\'echet spaces, viewed as a represenation of the group $H\times H$. 


 
\end{proof}












\begin{multicols}{2}[\section{Other chapters}]
\noindent
Preliminaries
\begin{enumerate}
\item \hyperref[introduction-section-phantom]{Introduction}
\item \hyperref[conventions-section-phantom]{Conventions}
\item \hyperref[fdl-section-phantom]{GNU Free Documentation License}
\item \hyperref[index-section-phantom]{Auto Generated Index}
\end{enumerate}
\end{multicols}



\bibliography{my}
\bibliographystyle{amsalpha}

\end{document}
