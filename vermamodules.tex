\IfFileExists{stacks-project.cls}{%
\documentclass{stacks-project}
}{%
\documentclass{amsart}
}

% The following AMS packages are automatically loaded with
% the amsart documentclass:
%\usepackage{amsmath}
%\usepackage{amssymb}
%\usepackage{amsthm}

\usepackage{amssymb}

% For dealing with references we use the comment environment
\usepackage{verbatim}
\newenvironment{reference}{\comment}{\endcomment}
%\newenvironment{reference}{}{}
\newenvironment{slogan}{\comment}{\endcomment}
\newenvironment{history}{\comment}{\endcomment}

% For commutative diagrams you can use
% \usepackage{amscd}
\usepackage[all]{xy}

% We use 2cell for 2-commutative diagrams.
\xyoption{2cell}
\UseAllTwocells

% To put source file link in headers.
% Change "template.tex" to "this_filename.tex"
% \usepackage{fancyhdr}
% \pagestyle{fancy}
% \lhead{}
% \chead{}
% \rhead{Source file: \url{template.tex}}
% \lfoot{}
% \cfoot{\thepage}
% \rfoot{}
% \renewcommand{\headrulewidth}{0pt}
% \renewcommand{\footrulewidth}{0pt}
% \renewcommand{\headheight}{12pt}

\usepackage{multicol}

% For cross-file-references
\usepackage{xr-hyper}

% Package for hypertext links:
\usepackage{hyperref}

% For any local file, say "hello.tex" you want to link to please
% use \externaldocument[hello-]{hello}
\externaldocument[introduction-]{introduction}
\externaldocument[representationtheory-]{representationtheory}
\externaldocument[representations-compact-]{representations-compact}
\externaldocument[liegroups-general-]{liegroups-general}
\externaldocument[liestructure-]{liestructure} 
\externaldocument[reductiveforms-]{reductiveforms}
\externaldocument[vermamodules-]{vermamodules}
\externaldocument[gKmodules-]{gKmodules}
\externaldocument[asymptotics-]{asymptotics}
\externaldocument[plancherel-]{plancherel}
\externaldocument[discreteseries-]{discreteseries}
%\externaldocument[algebraicgroups-]{algebraicgroups} 
%\externaldocument[harmonicanalysis-]{harmonicanalysis} 
%\externaldocument[automorphicforms-]{automorphicforms}
%\externaldocument[periods-]{periods}
%\externaldocument[traceformulalocal-]{traceformulalocal}
%\externaldocument[traceformulaglobal-]{traceformulaglobal}
%\externaldocument[arithmetic-]{arithmetic}
%\externaldocument[geometric-]{geometric}
\externaldocument[fdl-]{fdl}
\externaldocument[index-]{index}

% Theorem environments.
%
\theoremstyle{plain}
\newtheorem{theorem}[subsection]{Theorem}
\newtheorem{proposition}[subsection]{Proposition}
\newtheorem{lemma}[subsection]{Lemma}

\theoremstyle{definition}
\newtheorem{definition}[subsection]{Definition}
\newtheorem{example}[subsection]{Example}
\newtheorem{exercise}[subsection]{Exercise}
\newtheorem{situation}[subsection]{Situation}

\theoremstyle{remark}
\newtheorem{remark}[subsection]{Remark}
\newtheorem{remarks}[subsection]{Remarks}

\numberwithin{equation}{subsection}

% Macros
%
\def\lim{\mathop{\rm lim}\nolimits}
\def\colim{\mathop{\rm colim}\nolimits}
\def\Spec{\mathop{\rm Spec}}
\def\Hom{\mathop{\rm Hom}\nolimits}
\def\SheafHom{\mathop{\mathcal{H}\!{\it om}}\nolimits}
\def\SheafExt{\mathop{\mathcal{E}\!{\it xt}}\nolimits}
\def\Sch{\textit{Sch}}
\def\Mor{\mathop{\rm Mor}\nolimits}
\def\Ob{\mathop{\rm Ob}\nolimits}
\def\Sh{\mathop{\textit{Sh}}\nolimits}
\def\NL{\mathop{N\!L}\nolimits}
\def\proetale{{pro\text{-}\acute{e}tale}}
\def\etale{{\acute{e}tale}}
\def\QCoh{\textit{QCoh}}
\def\Ker{\text{Ker}}
\def\Im{\text{Im}}
\def\Coker{\text{Coker}}
\def\Coim{\text{Coim}}

\def\eqref #1{(\ref{#1})}


% OK, start here.
%
\begin{document}

\title{Verma modules and the category $\mathcal O$.}


\maketitle

\phantomsection
\label{section-phantom}


\tableofcontents

We need at least one reference \cite{reference} in each chapter.



\section{Verma modules}
We have seen\footnote{In class, but not in the notes!} that finite-dimensional representations of semisimple Lie algebras are completely reducible, and the irreducibles are generated by a highest weight vector (with dominant, integral weight). We now want to \emph{construct} those irreducible representations (in particular, to show that there is a unique one up to unique isomorphism for each given weight), and to compute their \emph{characters}.

In specific cases one can do that ``by hand'', constructing first the irreducible representations fundamental weights, and then the rest by taking tensor products of those (and subtracing copies of the representations already constructed). For instance, for $\mathfrak{sl}_n$ the $n-1$ fundamental representations are the first $n-1$ exterior powers of the standard, $n$-dimensional represntation.

For a more systematic approach, it is better to move outside the realm of finite-dimensional representations, constructing the universal objects with highest weight.

More precisely, we consider the category of $\mathfrak g$-modules of arbitrary, possibly infinite, dimension (no topology), and for $\lambda\in \mathfrak h^*$ (where $\mathfrak h$ denotes a universal Cartan, later to be identified with a Cartan subgroup of $\mathfrak g$ we let $M_\lambda$ denote the module with the universal property that for any $\mathfrak g$-module:
$$\Hom_{\mathfrak g}(M_\lambda, V) = \Hom_{\mathfrak b} (\mathbb C_\lambda,V).$$
The module $\mathfrak M_\lambda$ is called the \emph{Verma module} of weight $\lambda$, and it is very easy to see that it exists, namely:
$$M_\lambda = U(\mathfrak g)\otimes_{U(\mathfrak b)} \mathbb C_{\lambda-\rho}.$$

Notice that, by the PBW theorem, as a $\mathfrak b^-$-module:
\begin{equation}\label{Vermastructure}M_\lambda = U(\mathfrak n^-)\otimes_{\mathbb C} \mathbb C_{\lambda-\rho},
\end{equation}
where $U(\mathfrak n^-)$ acts by left multiplication on the first factor, and the $\mathfrak h$-action is the tensor product of the adjoint representation and the representation on $\mathbb C_{\lambda-\rho}$. (The tensor product of Lie algebra representations is defined as: $X(v\otimes w) = (Xv)\otimes w+ v\otimes (Xw)$.)

Therefore:
\begin{lemma}
 \begin{enumerate}
  \item $M_\lambda$ is $\mathfrak h$-locally finite and semisimple. The ($\mathfrak h$-)weights of $M_\lambda$ are of the form $(\lambda-\rho) - \sum_i c_i\alpha_i$, where $\alpha_i$ range over simple positive roots (we will denote their set by $\Delta$) and $c_i\in \mathbb Z$. The weight spaces are finite-dimensional, and $M_\lambda^{\lambda-\rho}$ is one-dimensional.
 \item $M_\lambda$ is $\mathfrak n$-locally finite.
 \end{enumerate}
\end{lemma}

\begin{proof}
 The first statement follows immediately from the presentation (\ref{Vermastructure}), and the second from the first and the fact that the action of $\mathfrak n$ raises weights.
\end{proof}

\begin{proposition}\label{uniquequotient}
 $M_\lambda$ has a unique irreducible quotient, which will be denoted by $L_\lambda$. 
\end{proposition}

\begin{proof}
 The sum of all proper submodules does not meet $M_\lambda^{\lambda-\rho}$, and hence is proper.
\end{proof}

\section{The category $\mathcal O$.}

The full subcategory of $\mathfrak g$-modules which are:
\begin{itemize}
 \item $\mathfrak h$-locally finite and semisimple;
 \item $\mathfrak n$-locally finite;
 \item finitely generated
\end{itemize}
is called category $\mathcal O$ (from a Russian paper of Gelfand-Gelfand-Bernstein). As we have seen, it contains Verma modules. %We will see that those are actually the basic building blocks of this category, more precisely they 

\begin{lemma}
A submodule of a module in $\mathcal O$ is in $\mathcal O$. The category is \emph{noetherian}, i.e.\ every increasing chain of subobjects of a given object stabilizes.
\end{lemma}

\begin{proof}
For the first statement, only finite generation is not obvious, but it follows from the fact that $U(\mathfrak g)$ is noetherian (a proposition of PBW).

The union of a chain of submodules is a submodule, hence finitely generated, hence the chain has to stabilize.
\end{proof}

We will eventually see that it is also Artinian, i.e.\ every object is of finite length.

\begin{lemma}\label{filtrationVerma}
Every object in $\mathcal O$ has a filtration whose quotients are surjective images of Verma modules.
\end{lemma}

\begin{proof}
 Let $V$ be in $\mathcal O$, and let $W\subset V$ be a finite-dimensional, generating subspace. Without loss of generality, $W$ is $\mathfrak b$-stable (for $U(\mathfrak b)W$ is, in any case, finite-dimensional). By Lie's theorem, it has a filtration with one-dimensional quotients. Therefore, $V$ has a filtration with quotients generated by $\mathfrak b$-eigenvectors. Each such representation is the surjective image of a Verma module.
\end{proof}

The \emph{Grothendieck group} of an abelian category $\mathcal C$ is the free group on its objects, modulo the relation: $[B]=[A]+[C]$ for every short exact sequence $0\to A \to B\to C\to 0$. We will eventually see that the Grothendieck group of $\mathcal O$ is generated by Verma modules, in fact: it is free on the set of Verma modules.


\section{The case of $\mathfrak{sl}_2$, and application.}

Here we identify the elements of $\mathfrak h^*$ with integers, according to their value on $\text{ch}eck\alpha$. Under this, $\rho$ corresponds to $1$, and therefore $M_\lambda$ denotes the Verma module with heighest weight $\lambda-1$. 

\begin{lemma}
 $M_\lambda$ is irreducible, unless $\lambda\in \mathbb Z_{>0}$, in which case there is an exact sequence:
$$ 0 \to M_{-\lambda}\to M_\lambda \to L_\lambda \to 0.$$
\end{lemma}

\begin{proof}
 Every submodule must have a highest weight vector, which must be of the form $F^nv_{\lambda-\rho}$. We compute that:
$$ EF^n v_{\lambda-\rho} = n (\lambda-n) F^{n-1} v_{\lambda-\rho},$$
therefore for it to be zero (for some $n>0$ we must have $\lambda\in \mathbb Z_{>0}$.
\end{proof}

We return to the case of a general semisimple $\mathfrak g$. Then:

\begin{lemma}
 If $\alpha$ is a simple root such that $\left<\lambda,\text{ch}eck\alpha\right>\in \mathbb Z_{>0}$ then there is an embedding: $M_{s_\alpha\lambda}\hookrightarrow M_\lambda$. The quotient $V=M_\lambda/M_{s_\alpha\lambda}$ has the property that it is locally $(\sl_2)_\alpha$-finite, where $(\sl_2)_\alpha$ denotes the embedding of $\sl_2$ into $\mathfrak g$ determined by the root $\alpha$.
\end{lemma}

\begin{proof}
 As in the previous lemma, we calculate that there is a highest weight vector with weight $s_\alpha\lambda$, hence there is a non-trivial map: $M_{s_\alpha\lambda}\to M_\lambda$. Since $M_{s_\alpha\lambda}, M_\lambda \simeq U(\mathfrak n^-)$ as $U(\mathfrak n^-)$-modules, and $U(\mathfrak n^-)$ does not have zero divisors, such a map has to be injective.

 With notation $(H_\alpha,E_\alpha,F_\alpha)$ for the $\sl_2$-triple corresponding to $\alpha$, we need to show that the quotient is $F_\alpha$-locally finite. (Finiteness under the other two is automatic for the category $\mathcal O$.) If $V'$ is the set of $F_\alpha$-finite vectors, then $V'\ni v_{\lambda-\rho}$; we claim that $V'$ is $\mathfrak g$-stable. Indeed, we have a homomorphism of $F_\alpha$-modules: $\mathfrak g\otimes V'\to V$, where $F_\alpha$ acts on $\mathfrak g$ via the adjoint representation. But $\mathfrak g$ is $F_\alpha$-finite and $V'$ is $F_\alpha$-locally finite, hence their tensor product is locally finite, therefore $\mathfrak gV'\subset V'$. Together with $v_{\lambda-\rho}\in V'$, this implies that $V'=V$.
\end{proof}


We haven't defined characters yet, but here's a proposition about them:
\begin{proposition}\label{characterWinv}
 For every subquotient of $V$, the character is $s_\alpha$-stable.
\end{proposition}
This says, in particular, that the set of weights of that subquotient is $s_\alpha$-stable.

The proposition follows from the theory of finite-dimensional $\sl_2$-representations. It implies:
\begin{proposition}
 Assume that $\lambda$ is \emph{integral} (i.e.\ $\left<\text{ch}eck\alpha,\lambda\right>\in \mathbb Z$ for all roots $\alpha$) and \emph{strictly dominant} (i.e.\ $\left<\text{ch}eck\alpha,\lambda\right>>0$ for all positive roots $\alpha$). Then the represenation:
$$L_\lambda' = M_\lambda/\left(\sum M_{s_\alpha\lambda}\right)$$
(sum over simple positive roots) is finite dimensional. In particular, $L_\lambda$ (the unique irreducible quotient of $M_\lambda$) is an (the) irreducible finite-dimensional representation with highest weight $\lambda-\rho$.
\end{proposition}

\begin{proof}
By the previous proposition, the quotient will have a $W$-stable set of weights. On the other hand, all weights are $\le \lambda$ and differ from $\lambda$ by an element of the root lattice, so there is a finite set of weights only. Finally, the weight spaces are finite dimensional, so the quotient is finite-dimensional.
\end{proof}

We will eventually see that $L_\lambda'=L_\lambda$. In particular, for each dominant integral weight $\lambda$ the \emph{exists} a (unique up to isomorphism) finite-dimensional representation $V_\lambda:= L_{\lambda+\rho}$ of $\mathfrak g$. (The uniqueness was a proposition of Proposition \ref{uniquequotient}.)



\section{Localization with respect to $\mathfrak z(\mathfrak g)$}

Recall that $\mathfrak z(\mathfrak g)$ denotes the center of the universal enveloping algebra. Lemma \ref{filtrationVerma} implies:

\begin{lemma}
$\mathfrak z(\mathfrak g)$ acts by scalars on Verma modules.  For every object $V$ in $\mathcal O$, the action of $\mathfrak z(\mathfrak g)$ on $V$ is locally finite.
\end{lemma}

\begin{proof}
 For the first statement, notice that $\mathfrak z(\mathfrak g)$ preserves weight spaces, so it must act by a scalar on the one-dimensional highest weight space. But that generates the whole module, so it acts by the same scalar on it.

 We have seen in Lemma \ref{filtrationVerma} that every object can be filtered by surjective images of Verma modules. The center acts by a scalar on a Verma module, hence on its quotients. Therefore it acts locally finitely on finite extensions of such objects.
\end{proof}



The following will be proven later:

\begin{theorem}[Harish-Chandra isomorphism]
There is an isomorphism of algebras $\phi:\mathfrak z(\mathfrak g)=\mathbb C[\mathfrak h^*]^W$, with the property that every $\Delta\in \mathfrak z(\mathfrak g)$ acts on the Verma module $V_\lambda$ by the scalar $\phi(\Delta)(\lambda)$.
\end{theorem}

Notice that the maximal ideals of $\mathbb C[\mathfrak h^*]^W$ are the complex points of the set-theoretic quotient $\mathfrak h^*/W$. Indeed, for every finite group $\mathbb G_amma$ acting on an affine variety $X$ the quotient $X// \mathbb G_amma$ is an affine variety (i.e.\ $k[X]^\mathbb G_amma$ is finitely generated), the map $X\to X// \mathbb G_amma$ is finite, and the quotient is also the so-called \emph{geometric quotient} (usually denoted $X/\mathbb G_amma$, although this can be mistaken for the \emph{stack-theoretic} quotient, which is the most ``correct'' one), in particular its points are in bijection with $G$-orbits on $X$. 

\begin{proposition}
 \begin{enumerate}
  \item The category $\mathcal O$ is a direct sum of categories $\mathcal O_\text{ch}i$, with $\text{ch}i$ varying over the complex points of $\mathfrak h^*/W$.
  \item If $\lambda$ is such that $\lambda-w\lambda$ is never a sum of the form $\sum_\alpha n_\alpha \alpha$, with $\alpha$ varying over simple (positive) roots and $n_\alpha\in\mathbb N$, then $M_\lambda$ is irreducible.
  \item Every object in $\mathcal O$ is of finite length. 
  \item The classes of the Verma modules $M_\lambda$ (or, equivalently, their irreducible quotients $L_\lambda$) are a basis for the Grothendieck group $\mathbb Z[\mathcal{O}]$.
 \end{enumerate}
\end{proposition}

\begin{proof}
 \begin{enumerate}
  \item Since the action of the center is locally finite, we can decompose every object into a direct sum of generalized $\mathfrak z(\mathfrak g)$-eigenspaces. Obviously, there are no $\mathfrak g$-morphisms between them.
  \item If $M_\lambda$ is not irreducible, there is a nontrivial map from $M_\mu$ to $M_\lambda$ for some $\mu$. But this can happen only if $\lambda-\mu$ is in the positive root monoid, and by the decomposition of categories it can only happen if $\mu=w\lambda$ for some $w\in W$.
  \item By Lemma \ref{filtrationVerma}, it suffices to show that Verma modules are of finite length. Define $K_\lambda$ by the short exact sequence:
$$ 0\to K_\lambda\to M_\lambda\to L_\lambda\to 0,$$
if nonzero then $K_\lambda$ admits a filtration as in Lemma \ref{filtrationVerma}, whose factors are surjective images of modules $M_\mu$ with $\mu < \lambda$. But by the decomposition of categories, $\mu$ has to be a $W$-conjugate of $\lambda$, and hence in a finite number of steps we will arrive at weights $\mu$ as in the previous statement, hence $M_\mu$ irreducible.
  \item Same argument, by induction on $\lambda$ (the starting point of the induction being the $M_\lambda$ of 2. The fact that the $L_\lambda$ also form a basis follows from the fact that the category is Artinian, and they are the only irreducible objects (non-isomorphic to each other). 
 \end{enumerate}
\end{proof}


\begin{proposition}
 When $\lambda$ is integral and strictly dominant, the representations $L_\lambda'$ of the previous subsection are irreducible (hence equal to $L_\lambda$).
\end{proposition}

 
\section{Characters}

Consider the ring $R$ of formal sums $\sum_{\lambda\in \mathfrak h^*} c(\lambda) e^\lambda$, where $c_\lambda$ is supported in a finite number of translates of the negative root monoid. Elements of the ring are multiplied as the notation suggests.

The \emph{character} $\text{ch}_V$ of an object $V$ in the category $\mathcal O$ (or a sub-$\mathfrak h$-module) is the following element of $R$: $\sum_{\lambda \in \mathfrak h^*} \dim V_\lambda e^\lambda$. It defines a group homomorphism from $\mathbb Z[\mathcal O]$ (the Grothendieck group of $\mathcal O$) to $R$. Moreover, it can easily be shown that the character of the tensor product of two representations is the product of the two characters. (The tensor product is not necessarily in $\mathfrak O$, but it has weight spaces such that this statement makes sense.)

Let $L = \prod_{\alpha>0} \left( e^\frac{\alpha}{2} - e^{-\frac{\alpha}{2}}\right) \in R$ (the product over all positive roots). Notice that $L$ can also be written as: $ e^\rho \prod_{\alpha>0} \left( 1 - e^{-{\alpha}}\right), $
hence it is supported on \emph{integral} weights. (The weight $\rho$ is integral because $\alpha = \rho- s_\alpha\rho = \left<\rho,\text{ch}eck\alpha\right> \alpha$ for every simple root $\alpha$.)

\begin{proposition}
 The character of the Verma module $M_\lambda$ satisfies:
$$ L\cdot \text{ch}i_{M_\lambda} = e^\lambda.$$
\end{proposition}

\begin{proof}
 As an $\mathfrak h$-module, $V_\lambda = U(\mathfrak n_-) \otimes \mathbb C_{\lambda-\rho}$, so $\text{ch}(V) = \text{ch}(U(\mathfrak n_-))\cdot \text{ch}(\mathbb C_{\lambda-\rho}) = \text{ch}i(U(\mathfrak n^-))\cdot e^{\lambda-\rho}$. Therefore, it suffices to prove that the character of $U(\mathfrak n_-)$ is the power series $e^\rho/L$.

 By the Poincare-Birkhoff-Witt theorem, as $\mathfrak h$ representations we have: $U(\mathfrak n_-) = \otimes_{\alpha>0} S^\bullet \mathfrak g_{-\alpha}$. The character of $S^\bullet \mathfrak g_{-\alpha}$ is $1+e^{\alpha}+e^{2\alpha}+\dots = \frac{1}{1-e^{-\alpha}}$, and this proves the proposition.
\end{proof}

Finally, we are ready to prove the \emph{Weyl character formula}:

\begin{theorem}
 The character of the irreducible representation with heighest weight $\lambda$ is given by the \emph{Schur polynomial}:
$$ \text{ch}(V_\lambda) = \frac{\sum_{w\in W} \text{sgn}(w) e^{w(\lambda+\rho)}}{L}$$
\end{theorem}

\begin{proof}
 Since in the Grothendieck group we have: $$[V_\lambda]= [M_{\lambda+\rho}] + \sum_{w\in W, w\ne 1} c_w [M_{w(\lambda+\rho)}],$$
we get:
$$ L\cdot \text{ch}(V_\lambda) = e^{\lambda+\rho} + \sum_{w\in W,w\ne 1} c_w e^{w(\lambda+\rho)}.$$

On the other hand, we know that the character should be $W$-invariant (Proposition \ref{characterWinv}), therefore the expression on the right should be $(W,\text{sgn})$-equivariant. Therefore, $c_w = \text{sgn}(w)$.
\end{proof}



%********************************************************************************************

\section{The Chevalley and Harish-Chandra isomorphisms}

adfadfa



\begin{multicols}{2}[\section{Other chapters}]
\noindent
Preliminaries
\begin{enumerate}
\item \hyperref[introduction-section-phantom]{Introduction}
\item \hyperref[conventions-section-phantom]{Conventions}
\item \hyperref[fdl-section-phantom]{GNU Free Documentation License}
\item \hyperref[index-section-phantom]{Auto Generated Index}
\end{enumerate}
\end{multicols}



\bibliography{my}
\bibliographystyle{amsalpha}

\end{document}
